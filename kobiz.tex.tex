פיתוח גלאי MWPC מבוסס אוויר כלוא לזיהוי חלקיקי אלפא (1–10 MeV) 1. מבוא ומודלים תאורטיים קיימים לגלאים פרופורציונאליים גלאי פרופורציונלי רב-חוטי (Multi-Wire Proportional Chamber – MWPC) הוא סוג של גלאי גז המסוגל לזהות חלקיקים טעונים וקרינה ולתת מידע כמותי על האנרגיה שלהם, ולעיתים אף את מיקומם במרחב en.wikipedia.org en.wikipedia.org . גלאי ה-MWPC הומצא בשנת 1968 על-ידי ג'ורג' שארפק ב-CERN, ומהווה פיתוח משופר של הגלאים הפרופורציונליים הראשונים en.wikipedia.org . יחודו של MWPC הוא שימוש במערך צפוף של חוטי אנודה דקים תחת מתח גבוה בתוך תא מלא גז, כך שכל חלקיק מיינן שחודר לתא יגרום לאירוע יינון שאת אותותיו ניתן להגביר ולזהות בכל אחד מהחוטים. בכך, MWPC משיג קצב ספירה גבוה בהרבה מבעבר (לעומת שיטות כמו תא הבועות ההיסטורי) ומאפשר לקבוע היכן עבר החלקיק לפי אילו חוטים זיהו אות en.wikipedia.org en.wikipedia.org . גלאים פרופורציונליים לעומת גלאים אחרים: גלאי גז יכולים לפעול בכמה תחומים של מתחים חשמליים, הנבדלים באופן הפעולה שלהם. להלן ההבדלים העיקריים בין סוגי הגלאים השונים: תא יינון (Ionization Chamber): פועל במתח נמוך יחסית, כך שהגז בתא אינו גורם להגבר מטען. כל זוג יונים הנוצר מחלקיק נכנס נאסף כשדה החשמל דוחף את האלקטרונים לאנודה והיונים החיוביים לקתודה. הזרם או המטען הנמדד פרופורציונלי ישירות לכמות היונים הראשונית. היתרון הוא שהאות לינארי לאנרגיית הקרינה, אך החיסרון הוא שאותי היציאה חלשים מאוד. לא ניתן כמעט להבדיל בין סוגי קרינה שונים באמצעות תא יינון, משום שכל האנרגיה נפרקת ישירות בלי הגברה – אלפא, בטא וגמא כולם יניבו אות חלש יחסית, ההבדל יהיה רק בכמות היונים (למשל אלפא, בהיותו חלקיק כבד וטעון, מיינן יותר מבטא וגמא ולכן יוצר זרם גדול מעט יותר, אך עדיין ללא אפשרות ברור להבחין ביניהם) nuclear-power.com . תא יינון דורש אלקטרוניקה רגישה במיוחד (אלקטרומטר) כדי למדוד את זרמי היונים המזעריים. גלאי פרופורציונלי (כמו MWPC או גלאי חוט בודד): פועל בתחום מתח בינוני שבו שדה חשמלי גבוה מספיק לגרום להגבר גזי אבל לא גבוה עד כדי פריקה מלאה. בגלאי פרופורציונלי, כל אלקטרון ראשוני שנוצר ביינון הגז יכול לקבל מספיק אנרגיה מהשדה כדי לגרום יינון משני – תהליך של מפולת טאונסנד (Townsend Avalanche) – וכך מתקבל הגבר גזי: מכל זוג יונים ראשוני מתקבלים מספר רב של יונים בסך הכל, וזרם גדול יותר. ההגבר קבוע ופרופורציונלי לכמות היונים הראשונית, ולכן אות המוצא עדיין נשאר פרופורציונלי לאנרגיית החלקיק. זה מאפשר לא רק לזהות שהייתה פגיעה אלא גם להבדיל בין חלקיקים לפי גודל הפולס (למשל פולס מאלפא גדול מפולס מבטא בשל ההבדל ביינון הראשוני) nuclear-power.com . בדרך זו אפשר לבצע ספקטrosopia (מדידת אנרגיה) חלקית בגלאי גז. מקדם ההגבר (Gas Gain) בגלאי פרופורציונלי טיפוסי נע בין $10^3$ ל-$10^4$ (פי אלף עד עשרת-אלפים) nuclear-power.com , מה שמשפר מאוד את יחס האות-לרעש לעומת תא יינון ומקל על דרישות המגבר האלקטרוני nuclear-power.com . גלאי גייגר-מילר (Geiger–Müller): פועל במתחים גבוהים עוד יותר, עד כדי כך שכל יון ראשוני יוצר מפולת אלקטרונים בלתי נשלטת לאורך כל נפח הגלאי. מתקבלת פריקה מלאה של הגז (פורקן) כך שאות הפולס רווי ואינו תלוי כלל באנרגיית החלקיק (כל אירוע נותן פולס זהה בגודלו). במצב זה אין יכולת להבדיל בין חלקיקים שונים או אנרגיות שונות – כל פולס מעיד רק על "הייתה קרינה". בגלאי GM מקדם ההגברה אפסי למעשה חסר גבול מעשי (עד $10^{10}$ בערך) nuclear-power.com , אך יש צורך ב"המת נהילה" (quenching) ובזמן מת (dead time) אחרי כל ספירה עד לשיקום הגז. GM טוב לספירת אירועים אך לא לספקטרוסקופיה או מדידות מדויקות של אנרגיה. MWPC בהקשר זה: ה-MWPC הוא למעשה צורה מורחבת של גלאי פרופורציונלי, שבה מספר רב של חוטי אנודה משמשים כמערכת חישה מרחבית. בעוד גלאי פרופורציונלי קלאסי מכיל אנודה יחידה (למשל חוט דק במרכז צילינדר קתודה), ב-MWPC ישנה מטריצה של חוטי אנודה המתוחים במקביל במרחקים שווים בתוך תא שטוח יחסית. הדבר מאפשר כיסוי שטח גדול ואפשרות לקבוע את מיקום החלקיק הפוגע לפי החוט או קבוצת החוטים שיוצרים פולס. כך MWPC משמש גם כגלאי מיקום (Tracker) ולא רק כמונה קרינה. בנוסף, ריבוי החוטים מגדיל את ההסתברות שלכל חלקיק תהיה אנודה קרובה מספיק למסלולו כדי לאסוף את האלקטרונים מהיינון. עם זאת, ריבוי החוטים מוסיף מורכבות אלקטרונית (ריבוי ערוצי מדידה) ואילוצי תכנון מכניים וחשמליים (מניעת ניצוצות ותפקוד יציב של כל החוטים יחד). כאשר חלקיק טעון (כגון אלפא) עובר דרך הגלאי, הוא מיינן את אטומי הגז לאורך מסלולו ומשאיר שובל של יוני גז: אלקטרונים חופשיים ויונים חיוביים. בתנאי שדה חשמלי, האלקטרונים נמשכים לכיוון האנודה (+) והיונים החיוביים נמשכים לקתודה (–). בגלאי MWPC, האנודות הן החוטים הדקים תחת מתח גבוה חיובי, והקתודות יכולות להיות קירות התא המתכתיים או יריעות מתכת במרחק מה מהחוטים. השדה החשמלי בתא מעוצב כך שימנע רקומבינציה (שילוב חזרה) של היונים, וימשוך את האלקטרונים מהר אל חוטי האנודה לפני שיילכדו בתהליכי צימוד כימי (עניין חשוב במיוחד אם יש גזים אלקטרושליליים כמו חמצן, שיכולים "ללכוד" אלקטרונים איטיים ולהפוך אותם ליוני שליליים). קרוב מאוד לחוטי האנודה – במרחק זעיר של חלקי מילימטר – השדה חזק מספיק כדי שאותם אלקטרונים נרכשים יוכלו ליינן אטומים נוספים וליצור מפולת אלקטרונים (מפולת טאונסנד) en.wikipedia.org nuclear-power.com . בתהליך מפולת טאונסנד, אלקטרון בודד מקבל מהשדה האנרגיה קינטית, מתנגש באטום גז ויוצר יון חדש (אלקטרון נוסף משתחרר). כעת שני האלקטרונים נעים לכיוון האנודה, וכל אחד יכול בהתנגשות הבאה לשחרר אלקטרון נוסף, וכך הלאה – מספר האלקטרונים גדל באופן מעריכי לאורך מסלול קצר מאוד, עד ליצירת "ענן" אלקטרונים קרוב לחוט האנודה en.wikipedia.org . תרשים סכמטי של תהליך זה מוצג באיור 1. בסופו של דבר, מכל אלקטרון התחלתי שנוצר מהחלקיק, מתקבל מספר גדול $M$ של אלקטרונים הנאספים על החוט – זהו מקדם ההגבר הגזי (Gas Gain). במילים אחרות, מתקבלת זרימת זרם גדולה יותר בפרופורציה ישירה לכמות היינון הראשונית, ולכן הפולס החשמלי גדול יותר וקל למדידה. התנאי הקריטי הוא שהגלאי מתופעל בתחום הפרופורציונלי, כלומר שכל אירוע יינון ראשוני יצור מפולת אחת בלבד והמפולת תיעצר מאליה מיד לאחר איסוף האלקטרונים על האנודה nuclear-power.com . כך מובטח שהאות מכל אירוע נשאר פרופורציונלי לאירוע המקורי (ולא רווי כמו בגלאי GM). ערכי ההגבר כאמור יכולים להגיע לאלפים ועד עשרות אלפים לפני שנכנסים לאזור אי-פרופורציונלי (שמעליו מתחיל שינוי בהתנהגות והאות כבר לא בפרופורציה מדויקת לכמות היינון הראשונית) nuclear-power.com . איור 1: המחשה סכמטית של תהליך מפולת טאונסנד. אלקטרון חופשי (כחול) נע בשדה חשמלי חזק לכיוון האנודה, מתנגש באטום גז ויוצר יון חדש (נקודה צהובה – "אירוע יינון"). האלקטרון המשתחרר (מסומן בכתום) גם הוא מואץ וחוזר ומיינן אטומים נוספים. התהליך מכפיל את מספר האלקטרונים בכל מדרגה עד ליצירת מפולת (Avalanche) של אלקטרונים רבים הנעים לאנודה. תופעה זו מאפשרת הגברה פרופורציונלית של האות בכל אירוע קרינה בגלאי גז. בכדי להבין את תכונות ה-MWPC, יש לסקור תחילה את התהליכים הפיזיקליים הבסיסיים: יינון גז על-ידי חלקיקי אלפא, טווח חלקיקי אלפא בחומר, ועקרונות ההגבר הגזי (מפולת טאונסנד ומודל ההגבר של דיתורן). נפרט על כל אחד מהנושאים הללו להלן. 1.1 יינון של חלקיקי אלפא בגז חלקיקי אלפא (גרעיני הליום טעונים, $He^{2+}$) הנפלטים מדוגמאות רדיואקטיביות (כגון ^241^Am, ^226^Ra וכד') נושאים אנרגיה קינטית אופיינית של מספר MeV. למשל, ^241^Am פולט אלפא באנרגיה של כ-5.486 MeV. כאשר חלקיק אלפא עובר דרך חומר, הוא יוצר יינון צפוף מאוד – כלומר, חלקיק אלפא מאבד את אנרגייתו תוך יצירת מספר גדול של זוגות יון-אלקטרון במסלול קצר. באנרגיות אלו, אלפא הוא חלקיק כבד יחסית ובעל מטען +2e, ולכן קצב איבוד האנרגיה שלו במסלולו גבוה בהרבה משל חלקיקי בטא (אלקטרונים) או פוטוני גמא. למעשה, אנרגיית היינון הממוצעת הנדרשת ליצור זוג יונים באוויר היא כ-34 eV nuclear-power.com . מכאן שחלקיק אלפא של 5 MeV יכול ליינן בסדר גודל של $\displaystyle\frac{5 ,\text{MeV}}{34 ,\text{eV}} \approx 1.5\times10^5$ אטומים לאורך מסלולו, אם כל האנרגיה תושקע ביצירת יונים. גם אם רק חלק מהאנרגיה הולכת ליוניזציה ישירה (והשאר לערור אטומים, חום וכו'), מדובר בעשרות אלפים של זוגות יונים לכל אלפא יחיד. לשם השוואה, חלקיק בטא (אלקטרון) באותה אנרגיה יוצר פחות יונים כי מסלולו מפותל וארוך יותר (יינון מפוזר), ופוטון גמא באותה אנרגיה עשוי בכלל לחדור את התא עם יינון מועט (הוא מיינן עקב אינטראקציות משניות כגון פיזור קומפטון או יצירת זוגות, בסבירות נמוכה יותר). בכך אלפא "נוחה" לגילוי בגלאי גז: היא משאירה אות מובחן וגדול יחסית. טווח חלקיקי אלפא בחומר: עקב היינון הצפוף, חלקיקי אלפא מאבדים את אנרגייתם במהירות ומתוך כך טווח החדירה שלהם בחומרים הוא קצר. הטווח מוגדר כאורך המסלול עד לעצירה כמעט מוחלטת של החלקיק. באוויר (בלחץ אטמוספרי ובטמפרטורה חדר), טווח האלפא הוא בסדר גודל של סנטימטרים ספורים עבור אנרגיות בנות כמה MeV. למשל, אלפא של 5 MeV תעבור כ-3.5–4 ס"מ באוויר doeic.science.energy.gov , בעוד אלפא של 10 MeV יכולה להגיע לכ-10 ס"מ באוויר courses.washington.edu . אנרגיות נמוכות יותר, למשל 1 MeV, יניבו טווח קצר בהרבה – בערך כמה מילימטרים בודדים. את התלות בין אנרגיית אלפא לטווח באוויר מתאר בקירוב חוק אמפירי: $R \approx 0.32 E^{3/2}$ (כאשר $R$ בס"מ, $E$ ב-MeV, עבור אוויר בתנאי מעבדה ~15°C, 1atm) chegg.com . באיור 2 מוצגת עקומת טווח-אנרגיה משוערת של חלקיקי אלפא באוויר בתחום 0–10 MeV, לפי נוסחה זו. ניתן לראות, למשל, כי עבור $E=5$ MeV מתקבל $R\approx3.6$ ס"מ, ועבור $E=10$ MeV מתקבל $R\approx10$ ס"מ. נתונים אלו עולים בקנה אחד עם המדווח בספרות. חשוב לציין שטווח חלקיקי אלפא בחומרים צפופים יותר קצר בסדרי גודל: באותו אנרגיה של 5 MeV, אלפא יעבור רק עשרות מיקרונים ב"חומר צפוף" כמו רקמת אדם או מתכת doeic.science.energy.gov . משמעות הדבר היא שאלפא של 5 MeV, הניתנת לעצירה ע"י כמה ס"מ של אוויר, לא תצליח לחדור אפילו דף נייר עבה או מעט עור אנושי. עובדה זו משפיעה על תכנון חלון הכניסה לגלאי – יש להשתמש בחלון דק במיוחד אם רוצים לאפשר לחלקיקי אלפא חיצוניים להיכנס לתא (ראו פירוט על חלון הגלאי בהמשך). איור 2: יחס מקורב בין אנרגיית חלקיק אלפא לבין טווח החדירה שלו באוויר (בלחץ וטמפרטורה סטנדרטיים). האלפאיות (1–10 MeV) נבלמות לאחר סנטימטרים ספורים בלבד של אוויר. לדוגמה, אלפא 5 MeV עוברת כ־3.6 ס"מ, בעוד אלפא 10 MeV יכולה לעבור כ־10 ס"מ. ערכים אלו נובעים מקצב היינון הגבוה של האלפא, שגורם לאיבוד אנרגיה מהיר בחומר. בזמן שחלקיק האלפא נעצר בגלאי, האנרגיה הקינטית שלו מנותבת בעיקרה ליינון. חלק מהאנרגיה יכול ללכת לעירור אטומים (פליטת פוטוני אור/על-סגול) או לתגובות אחרות, אך רוב מוחלט מהאנרגיה של אלפא (כ-99%) מתבזבזת ביצירת יוני גז לאורך המסלול, עד שהחלקיק נעצר כליל. לכן, אם כל האנרגיה של חלקיק 5 MeV נבלעת בתוך נפח הגלאי, נקבל on the order of $10^5$ זוגות יונים, שכאמור יהפכו בעקבות ההגברה הגזית לכ-$10^8$–$10^9$ אלקטרונים שייאספו בחוט (תלוי במקדם ההגבר). בפועל, לא כל האנרגיה בהכרח תיבלע: ייתכן שהחלקיק ייעצר בדופן התא לפני שאיבד את כל האנרגיה, או שחלק מהאלקטרונים המשניים ילכדו. באפשרותנו לכייל את הגלאי כך ש"פסגת" הפולסים (distribution) תתאים לאנרגיית אלפא ידועה, ובכך לדעת שכל אלפא מלאה שנבלמה תניב פולס בגודל צפוי (ראו בהמשך תיאור הכיול עם מקור ^241^Am). 1.2 מפולת טאונסנד ומנגנון ההגברה הפרופורציונלית תהליך המפולת (Townsend Avalanche): כפי שתואר, כאשר אלקטרונים מגיעים לאזור שדה חזק ליד אנודת החוט, הם מתחילים להאיץ וצוברים אנרגיה בין התנגשויות. אם האנרגיה שהם צוברים עד להתנגשות הבאה עולה על פוטנציאל היוניזציה של אטום בגז, האלקטרון ייינן את האטום בהתנגשות וישחרר אלקטרון חדש. כעת שני האלקטרונים ממשיכים לעבר האנודה, ואם שדה החשמל מספיק חזק על פני המרחק עד ההתנגשויות הבאות – כל אחד מהם עשוי שוב ליינן אטומים נוספים. התוצאה היא גידול מעריכי במספר האלקטרונים כנגד כיוון התנועה, בדומה ל"ריבוי יציב" (chain reaction) המוגבל רק ע"י זמינות אטומים ומגבלות שדה. את קצב הגידול ניתן לתאר באופן רציף על-ידי מקדם יינון ראשון שמסומן בדרך-כלל $\alpha$ (אלפא): זהו מספר ההתנגשויות המייננות שהאלקטרון צפוי לבצע ליחידת אורך שהוא עובר. ערכו של $\alpha$ תלוי כמובן בעוצמת השדה $E$ ובצפיפות הגז (כלומר בלחץ), שכן שדה חזק ולחץ נמוך יתנו הסתברות גבוהה יותר ליינון (אלקטרון נע מהר יותר ומתנגש בפחות אטומים לאורך נתון). לעיתים מבטאים את $\alpha$ כיחס ליחידת לחץ: $\alpha/p = A \exp!\Big(-\frac{B p}{E}\Big)$, שם $A, B$ הן קבועים תלויי-גז (ניתנים בטבלאות עבור גזים שונים). צורה זו מבטאת שבשדות חזקים (או לחץ קטן) $\alpha$ גבוה – כלומר הרבה יינונים ליחידת אורך, ולעומת זאת בשדה חלש מתחת לסף מסוים לא יתרחשו כמעט יינונים ($\alpha \to 0$). עבור אוויר יבש בלחץ אטמוספירי, לדוגמה, ישנו סף שדה בסביבות $E/p \sim 24$ [V·cm^−1·Torr^−1] שמעליו מתחילה הכפלה משמעותית של אלקטרונים (זהו ערך $K$ בגז אוויר – ראו דיון במודל דיתורן בהמשך) mdpi.com . נגדיר $N_0$ אלקטרונים ראשוניים (למשל ממסלול אלפא) הנכנסים לאזור השדה החזק מסביב לאנודה. מספרם יגדל לפי $\displaystyle N(x) = N_0 e^{\int \alpha(x),dx}$ לאורך מסלולם לכיוון האנודה en.wikipedia.org . באזורים רחוקים מהאנודה $E$ קטן וייתכן ש-$\alpha=0$, אך קרוב מספיק לחוט (בסדר גודל $<1$ מ"מ) השדה גדל כך ש-$\alpha>0$ וגדל ככל שמתקרבים (שדה עולה $\implies$ $\alpha$ עולה). האינטגרל של $\alpha$ מהרדיוס שבו מתחילה המפולת ($r = r_0$ נקרא "רדיוס התחלתי") ועד רדיוס החוט ($r=a$) יתן את סך ההגבר: $\displaystyle M = \exp!\Big(\int_{r_0}^{a} \alpha(r),dr\Big)$. מכאן מוגדר מקדם ההגבר $M$ – מספר האלקטרונים הכולל שנאסף באנודה לכל אלקטרון ראשוני בודד. לדוגמה, אם $M=10^4$, פירושו שכל אלקטרון שמקורו ביינון ראשוני גורר יצירת 10,000 אלקטרונים בסה"כ במפולת. במערכת ללא רוויה, $M$ זהה לכל האלקטרונים הראשוניים (תחת אותם תנאים), ולכן אם היו $N_0$ אלקטרונים ראשוניים נקבל $N_0 \cdot M$ אלקטרונים בסה"כ – פרופורציה ישירה. מודל ההגבר של דיתורן: אנליזה קלאסית למיפת ההגבר בגלאי פרופורציונלי גלילי הציג דיתורן (J. E. Diethorn) בשנות ה-1950. המודל מניח ביטוי אקספוננציאלי למקדם היוניזציה $\alpha(E)$ כפי שתואר, ומשלב אותו בשדה החשמלי עבור גאומטריית חוט-וצילינדר. עבור גלאי בעל חוט אנודה דק (רדיוס $a$) בתוך צילינדר קתודה ברדיוס $b$, עם מתח $V_0$ בין האנודה לקתודה, השדה החשמלי כתלות במרחק $r$ מן החוט הוא: 𝐸 ( 𝑟 )    =    𝑉 0 𝑟   ln ⁡ ( 𝑏 / 𝑎 )   , E(r)= rln(b/a) V 0   , בהנחה של שדה סטטי רדיאלי (זו קירוב סביר כאשר $b \gg a$) uspas.fnal.gov . השדה חזק ביותר ליד החוט (כאשר $r \to a$). דיתורן הראה שמקדם ההגבר ניתן אז לחישוב באמצעות: ln ⁡ 𝑀    =    𝑉 0 − 𝐾 𝑝   𝑎   ln ⁡ ( 𝑏 / 𝑎 ) Δ 𝑉   , lnM= ΔV V 0  −Kpaln(b/a)  , כאשר $p$ הוא לחץ הגז, $K$ הוא ערך השדה המזערי המנורמל ללחץ (כפי שתואר לעיל) הדרוש לתחילת הגברה גזית, ו-$\Delta V$ היא האנרגיה הממוצעת שהאלקטרון רוכש מהשדה בין התנגשויות רצופות (תלויה בצפיפות ובחתכי הפעולה של הגז) mdpi.com mdpi.com . שני הפרמטרים $K$ ו-$\Delta V$ נקבעים ניסיונית עבור גזים שונים. למשל, עבור תערובת גז P-10 (90% ארגון + 10% מתאן) מקבלים $K\approx48.4$ [V·cm^−1·Torr^−1] ו-$\Delta V \approx 23.6$ eV mdpi.com . עבור גזים אחרים (ניאון, פחמן דו-חמצני, הליום, וכדומה) יש ערכים שונים. עבור אוויר יבש ערכי $K, \Delta V$ אינם מתפרסמים בטבלה הנ"ל, אך ניתן להניח שהם מסדר גודל דומה לתערובות גז עשירות בחנקן. חנקן (N_2) הוא המרכיב העיקרי באוויר (~78%), ומבחינת יינון ראשוני הוא דומה לארגון, אך קיים גם חמצן (~21%) שהינו גז אלקטרושלילי העלול לבלוע אלקטרונים איטיים. עובדה זו משפיעה על מקדם ההגבר: באוויר טהור נדרש שדה חזק מעט יותר להתחלת מפולת, משום שחמצן עלול "לגנוב" אלקטרונים לפני שיצליחו להגיע לאנודה. לכן $K$ עבור אוויר צפוי להיות מעט גבוה מזה של P-10 (יתכן סביב 50–60 ביחידות הנ"ל), ו-$\Delta V$ אולי שווה ערך לעשרות eV בודדות. ניתן לראות ממשוואת דיתורן כי: כאשר $V_0 = K p,a,\ln(b/a)$ מתקיים $\ln M = 0$, כלומר $M=1$. זהו סף המתח המינימלי שבו מתחילה הגברה גזית (מתח מתחת לסף זה לא יגרום למפולת כלל, והגלאי יתפקד רק כתא יינון). נסמן $V_{\text{min}} = K p,a,\ln(b/a)$. עבור $V_0 > V_{\text{min}}$, ההגבר עולה בערך באופן אקספוננציאלי עם המתח. בפרט, כל עליה של $\Delta V$ (כמה עשרות וולט) מעל הסף מביאה להכפלה של $M$ פי $e$ (בערך 2.718...). דוגמה מספרית: נניח חוט אנודה ברדיוס $a = 0.02$ מ"מ (20 מיקרון, גודל אופייני לחוטי טונגסטן מוזהבים המשמשים ב-MWPC) בתוך תא בעל קתודה במרחק $b \approx 10$ מ"מ (למשל חצי מרווח בין לוחות הקתודה, במקרה של MWPC שטוח – ראו תרשים מבנה בהמשך). עבור לחץ $p=1$ atm (אוויר כלוא), נקבל $\ln(b/a) = \ln(10\text{mm}/0.02\text{mm}) \approx \ln(500) \approx 6.21$. אם עבור אוויר ניקח $K \approx 50$ [V·cm^−1·Torr^−1] (הערכה), יש להמיר יחידות: $50$ V·cm^−1·Torr^−1 = $50 \times 1 ,\text{Torr}/1,\text{atm} \times 1,\text{cm}/0.01,\text{m} = 50 \times 760 \times 100 = 3.8\times10^6$ V·m^−1·Pa^−1 (כפולות להמרת טור ללחץ SI ומטר לס"מ). נכפיל: $K p,a = 3.8\times10^6 ,\text{V/m/Pa} \times 1.013\times10^5 ,\text{Pa} \times 2\times10^{-5},\text{m} \approx 770$ V. עכשיו $V_{\text{min}} = 770 \times 6.21 \approx 4780$ V. ערך זה נראה גבוה למדי, מה שמרמז שההערכה שלנו ל-$K$ עבור אוויר ייתכן גבוהה מדי או שהנחות המודל לא מדויקות לגאומטריה שלנו. אם ניקח $K$ מעט נמוך יותר (נניח 36 כפי שהוא לארגון/CO_2), יתקבל $V_{\text{min}} \approx 3450$ V. בפועל, ניסיונית ידוע שבאוויר בלחץ אטמוספירי, ניצוץ חשמלי (פריצה) יחל רק בסביבות 3–4 kV עבור אלקטרודות במרחק סנטימטר – זה מתיישב עם סדר הגודל שקיבלנו. ואכן, על מנת שיתחילו מפולות, יש צורך בקרבה לערך הפריצה. משמעות מעשית: כאשר עובדים עם אוויר בתור גז לגלאי פרופורציונלי, נדרשים מתחים גבוהים (אלפי וולטים) כדי להשיג הגבר משמעותי. כדי להקטין דרישה זו, משתמשים בחוטי אנודה דקים מאוד (מגדילים $\ln(b/a)$ על-ידי הקטנת $a$) ובמרווחי אנודה-קתודה קטנים (מורידים $b$). בדוגמה לעיל, $V_{\text{min}}$ ירד משמעותית עבור חוט של 10 מיקרון או עבור מרחק $b$ קטן מ-10 מ"מ. למשל, אם $a=0.01$ מ"מ, $V_{\text{min}}$ יפחת בערך בחצי (לכ-2400 V), ואם במקום 1 atm נעבוד בלחץ נמוך יותר – גם כן $V_{\text{min}}$ ירד פרופורציונלית (נניח בחצי אטמוספירה, מספיק כ-1200 V להתחלת הגבר). במילים אחרות, הנדסת הגלאי מכוונת להקטין ככל הניתן את מתח העבודה הדרוש להתחלת המפולות, וזאת באמצעות: חוטים דקיקים, מרחק קטן בין האנודה לקתודה, ולעיתים שימוש בתערובות גז רגישות (ערכי $K, \Delta V$ נוחים). במודל דיתורן, אם נציב $V_0$ גבוה בהרבה מסף $V_{\text{min}}$, נקבל $\ln M$ גדול מאוד. למשל, עבור $V_0 = 2 V_{\text{min}}$, מקבלים $\ln M = (V_0 - V_{\text{min}})/\Delta V \approx V_{\text{min}}/\Delta V$. אם $\Delta V \sim 30$ eV (כמה עשרות וולט), אז $\ln M \sim \frac{V_{\text{min}}}{30,\text{V}}$ שיוצא מס' עשרות עד מאות, כלומר $M$ אסטרונומי (עד $10^{50}$!). ברור שבמציאות זה לא אפשרי – לא ניתן להגביר מטען באין-סוף, ובוודאי לפני כן יתרחשו תהליכים אחרים שיעצרו את ההגבר. אכן, בניסויים נמצא שיש גבול עליון מעשי להגבר בגלאי פרופורציונלי, כאשר המפולות נעשות גדולות מדי ומתחיל להופיע אפקט שנקרא גבול ראתר (Raether limit): במפולת שמכילה סדר גודל של $10^7$–$10^8$ אלקטרונים, המטען הרב גורם לשדה עצמי ולתופעות כמו פליטת קרינה אולטרה-סגולה ויינון משני של הקתודה, והגלאי נכנס למצב פריצה (סטרימר, או למעשה מעבר למצב גייגר). לכן לעולם לא נפעיל בכוונה גלאי פרופורציונלי כך שיגיע להגברים גבוהים קיצונית. נהוג לעבוד בטווח הגבר נוח של עד $10^4$–$10^5$ לכל היותר nuclear-power.com , בו הגלאי עדיין יציב ופרופורציונלי. דוגמה לעקומת הגבר תאורטית לפי המודל (עבור גז P-10) מודגמת באיור 3 – רואים שמעל מתח מסוים הגרף מזנק בחדות, אך יש לזכור שבפועל מעבר לכ-$10^6$ הגבר הגלאי יסטה מתאוריה ויתקרב לסף פריצה. איור 3: עקומת מקדם הגבר הגז $M$ כתלות במתח האנודה $V_0$ לפי משוואת דיתורן (חישוב עבור חוט $a=20\ \mu$m, $b=10$ מ"מ, גז ארגון-מתאן 90:10 בלחץ 1 atm, עם $K, \Delta V$ כנתונים ידועים mdpi.com ). הקו המקווקו האדום מציין את סף המתח $V_{\min}$ שבו מתחילה הגברה ($M>1$). התאורייה צופה עלייה מעריכית של ההגבר עם המתח (שים לב לסקלת $M$ לוג-ערכית), אך במציאות יש גבול מעשי לגודל ההגבר לפני שהגלאי פורץ לניצוץ. 1.3 סיכום תאורטי – התנאים האידאליים לפעולת MWPC לאור המודלים התאורטיים, נוכל לסכם מהם התנאים והתכונות הדרושים לתכנון ובניית גלאי MWPC מבוסס אוויר, כך שיגלה באופן מיטבי חלקיקי אלפא בטווח 1–10 MeV: הגלאי צריך לפעול בתחום הפרופורציונלי, כלומר לספק הגבר גזי של $10^3$–$10^5$ לאות, אך להימנע מכניסה לאזור פריצה (סטרימר/גייגר). לכן יש לכוונן את מתח העבודה לגבול הנמוך שמאפשר גילוי יציב של אלפא. על מנת להשיג הגבר בהינתן שאוויר הוא הגז (הדורש שדה חזק למפולת), יש לעצב את הגאומטריה כך שהשדה המקסימלי יישאר גבוה: שימוש בחוטי אנודה דקים מאוד (20–50 מיקרון), מרווח קטן בין האנודה לקתודה (ס"מ ספורים), ולחץ גז לא גבוה מדי (אפשר לשקול תת-לחץ קל להורדת $V_{\text{min}}$). כמו כן, ניתן לשלב תערובת גזים (למשל הוספת גז "מרפה" כמו CO_2 או CH_4 באחוזים קטנים) כדי לשפר יציבות ולרדת בסף המתח, אך הדרישה שלנו דווקא אוויר כלוא מטעמי פשטות – משמע נסתפק באוויר ונעמוד באילוציו. חלקיקי אלפא בטווח 1–10 MeV כאמור נבלמים עד כ-10 ס"מ באוויר. מכיוון שעומק התא שלנו קטן (עד 2 ס"מ), יש לוודא שחלקיקי אלפא שאנו רוצים לגלות אכן עוצרים בתוך הגז, אחרת יפקדו חלק מהאנרגיה בדופן וייתכן שלא יהיו בעלי אות מלא. אפשר להשיג זאת ע"י הנחת מקור האלפא בתוך התא (כך שהאלפא תנוע בתוך הגז לכל היותר עד הדופן ממול). אם האלפא מגיעה מבחוץ, יש להכין חלון כניסה דקיק ביותר המאפשר מעבר אלפא עם איבוד מזערי, ולהבין שרק האלפא החזקים (עד ~5 MeV) אולי יחדור 2 ס"מ של אוויר לפני עצירה. אנרגיות נמוכות מ-2–3 MeV ייעצרו מוקדם יותר ועלולות שלא להגיע לאזור רגיש. פתרון אפשרי הוא למקם את חוטי האנודה קרוב ככל האפשר לחלון הכניסה, כך שגם אלפא חלש יגיע לסביבת האנודה לפני עצירה מלאה. בהינתן אלפא יוצרת מטען ראשוני גדול (עד $10^5$ זוגות יונים), אפילו הגבר של $10^3$ יניב $10^8$ אלקטרונים – מטען כולל של $\sim 1.6\times10^{-11}$ קולון (16 פיקו-קולון). זהו פולס גדול למדי שקל למדידה. לכן אין הכרח לכוון להגבר מקסימלי – די בהגבר ביניים שמספיק להתגבר מעל רעשי האלקטרוניקה. יתרה מזאת, הגבר גבוה מדי עלול להביא לאפקטים לא פרופורציוניים. לפיכך מיטבי לעבוד בהגבר "נמוך" ככל האפשר שעדיין יניב יחס אות-רעש טוב. באופן ניסיוני, סביר שמתחי עבודה בתחום 1200–1600 V יספיקו, בעוד שמתח 2000 V עלול לקרב לפריצה. לאחר שסקרנו את הרקע התאורטי, נעבור כעת לחישובים ולתכנון המפורט של הגלאי, בהתאם לעקרונות לעיל. 2. חישובים פיזיקליים מפורטים והשלכות על ביצועי הגלאי בפרק זה נערוך חישובים כמותיים להמחשת הגדלים האופייניים בגלאי, ונבחן דוגמאות קיצון מול ערכים מיטביים. כל נוסחה תלווה בהסבר מילולי, על-מנת להבהיר את משמעותה הפיזיקלית. החישובים יאפשרו לנו להבין את גבולות התכנון: אילו מתחים, גאומטריות ופרמטרים יבטיחו זיהוי אמין של חלקיקי אלפא (1–10 MeV) באוויר, ומה השפעת שינויים בפרמטרים על ביצועי הגלאי (רגישות, יעילות, רזולוציה וכדומה). 2.1 שדה חשמלי במערך אנודה-קתודה וחישוב מתחים נדרשים נתחיל בהערכת סדרי גודל של השדה החשמלי הדרוש בתוך הגלאי כדי לגרום להגבר גזי, וכן נחשב את המתח החשמלי שיש להפעיל כדי להגיע לשדה זה. כפי שראינו, עבור אוויר ערכי $E/p$ סביב $24$ [V·cm^−1·Torr^−1] (שהם כ-$3\times10^6$ V/m בלחץ atm) מתחילים את תהליך המפולת. עבור המרחקים הזעירים ליד חוט האנודה (עשרות מיקרונים), שדות של $10^6$–$10^7$ V/m הם ריאליים. נשאל: איזה מתח $V_0$ דרוש כדי לייצר שדה בסדר גודל זה ליד החוט? נשתמש בנוסחת השדה של חוט דק בתוך גליל (הנדסת MWPC מהווה קירוב לכך): $E(r) = \frac{V_0}{r,\ln(b/a)}$. את השדה הכי גבוה נקבל ממש על שטח פני החוט ($r=a$). לכן $E_{\text{max}} \approx \frac{V_0}{a,\ln(b/a)}$. נציב מספרים טיפוסיים: $a = 20$ מיקרון = $2\times10^{-5}$ m, $b = 10$ מ"מ = $1\times10^{-2}$ m, קיבלנו $\ln(b/a) \approx 6.21$. דרוש $E_{\text{max}} \sim 5\times10^6$ V/m (כדי לעבור את סף Townsend). מהמשוואה נקבל: 𝑉 0 ≈ 𝐸 max ⋅ 𝑎 ln ⁡ ( 𝑏 / 𝑎 ) = 5 × 10 6 [ V/m ] ⋅ 2 × 10 − 5 [ m ] ⋅ 6.21 ≈ 620  V . V 0  ≈E max  ⋅aln(b/a)=5×10 6 [V/m]⋅2×10 −5 [m]⋅6.21≈620 V. תוצאה זו מעט מטעה – היא מחשבת את המתח הדרוש ליצור $5\times10^6$ V/m ממש על החוט. אכן, כחישוב קודם קיבלנו ערך סף $V_{\text{min}} \approx 600$–$700$ V. אולם, יש לזכור שזהו החישוב עבור גז "נוח" (בדוגמה לעיל P-10) – באוויר $K$ כנראה גבוה יותר ולכן השדה הדרוש אף גדול יותר. אם נדרוש $E_{\text{max}} \approx 8\times10^6$ V/m, נקבל $V_0 \approx 1000$ V. זהו כבר סדר גודל שיותר תואם את ניסיוננו המעשי עם גלאי פרופורציונלי: בדרך-כלל כמה מאות וולטים לא מספיקים, ודרוש לפחות kV אחד כדי לראות פולסים. כלומר, אנו מצפים שמעגלי ההגבר יתחילו לראות פולסים בסביבות 800–1000 V, ושהתחום האופטימלי ינוע סביב 1200–1600 V. מתחת 800 V הגלאי יתפקד כתא יינון בלבד (יתכן שלא נבחין בפולסים מעל הרעש), ומעל 1600 V אולי ניכנס לאזור לא יציב. כמובן, ערכים אלה תלויים מאוד בגאומטריית הגלאי ובנוכחות גורמים כמו לחות או זיהומים שיכולים להשפיע על מתח הפריצה. נסקור גם את ההבדל בין גאומטריית חוט לבין לוחות שטוחים: בשדה בין לוחות במרחק $d$ זה מזה, נדרשת בערך $V \approx E\cdot d$ כדי לקבל שדה $E$. למשל, כדי לקבל $5\times10^6$ V/m במרווח 1 ס"מ, דרוש $5\times10^4$ V = 50 kV! זאת לעומת $~620$ V עבור חוט שקיבלנו לעיל. ההבדל הוא בחלוקת הפוטנציאל: בחוט דק, מרבית ירידת המתח היא קרוב לחוט (כי $E \propto 1/r$ גדל מאוד), בעוד בין לוחות יש שדה כמעט אחיד בכל המרווח. לכן חוטי אנודה דקים מאפשרים להשיג שדות עוצמתיים באזורים מקומיים בגלאי, ללא צורך במתחים עצומים. מצד שני, מחיר החוט הוא שהשדה אינו אחיד – ברוב נפח התא (רחוק מהחוטים) $E$ קטן יותר, ועשוי אף להיות חלש מכדי לאסוף את כל האלקטרונים. זהו שיקול בתכנון: נרצה מרווחי חוטים מספיק קטנים כדי שאף אלקטרון משוטט לא "ייתפס" באזור חלש ויילכד לפני שיגיע לחוט. נרחיב על כך כשנבחן את מרווחי החוטים (סעיף 7). 2.2 דוגמה חישובית: מספר היונים וגדלי האות כעת נחבר בין כמות היונים שנוצרת באירוע אלפא לבין האות החשמלי שיתקבל, כדי לוודא שהאות מספיק גדול למדידה ולא גדול עד כדי saturation: מספר זוגות היונים הראשוניים ($N_0$): כפי שחישבנו, אלפא 5 MeV יכולה ליצור בסדר $10^5$ זוגות. אם האנרגיה נמוכה יותר (למשל 2 MeV) או אם חלק ממסלולה מחוץ לגלאי, $N_0$ קטן בהתאם. נניח טווח מקרים: אלפא 1 MeV שעצרה בדיוק בגבול הגלאי – אולי רק $2\times10^4$ זוגות; אלפא 5 MeV שנעצרה במלואה – עד $1.5\times10^5$ זוגות; אלפא 10 MeV (אם באופן היפותטי היה נכנס ונעצר כולו בגלאי 2 ס"מ, מה שלא יקרה למעשה) – אולי קרוב ל-$3\times10^5$ זוגות. מקדם ההגבר ($M$): בבחירת מתח עבודה, נרצה $M$ בינוני. נניח למשל $M = 2000$ (הגבר $2\times10^3$). זה ערך שבהחלט אפשרי, אולי באזור 1200–1300 V לפי אומדן מוקדם. אם נעלה ל-1500 V, אולי $M=10^4$. מספר האלקטרונים הנאספים: $N_e = N_0 \cdot M$. עבור $N_0 = 10^5$ ו-$M=2000$ נקבל $2\times10^8$ אלקטרונים. עבור $N_0 = 2\times10^4$ ו-$M=2000$ נקבל $4\times10^7$ אלקטרונים. מטען כולל על האנודה: $Q = N_e \cdot e$ (כאשר $e = 1.602\times10^{-19}$ קולון, מטען האלקטרון). ממשיכים את הדוגמאות: $2\times10^8$ אלקטרונים הם $Q \approx 3.2\times10^{-11}$ C (32 פיקו-קולון). $4\times10^7$ אלקטרונים הם $6.4\times10^{-12}$ C (6.4 פיקו-קולון). מתח בפולט (Signal Voltage): כאשר מטען $Q$ מגיע למגבר הטעינה (שנפרט עליו בפרק האלקטרוניקה), הוא נטען על קבל הזנה $C_{\text{in}}$ ובכך מייצר שינוי מתח $\Delta V = Q/C$. קיבול המגבר טיפוסית בערכים של כמה עשרות פיקו-פאראד (נניח $C=30$ pF). לכן פולס של $32$ pC יניב $\Delta V \approx \frac{32\text{pC}}{30\text{pF}} \approx 1.07$ V. פולס של $6.4$ pC יניב $\sim0.21$ V. אלו מתחים ניכרים, שקל למדוד עם הגברה נוספת או ישירות בדיסקרימינטור. למעשה, גם מטען קטן יותר (נניח 1 pC) יתן $\sim33$ mV – עדיין מעל רמת הרעש ברב המערכות. סיכום ביניים: גם במקרה החלש יחסית (אלפא גבולית 1–2 MeV בהגבר 2000) נקבל פולס כ-0.2 V, ואילו במקרה העוצמתי (אלפא מלא 5 MeV בהגבר 10000) עלול לצאת פולס של כ-5 V ויותר. האלקטרוניקה צריכה להיות מתוכננת לטווח דינמי כזה כדי שלא לחתוך (לרוות) את הפולסים הגדולים, ומצד שני לזהות באופן ברור את הפולסים הקטנים ביותר הרצויים. זמן ועליית הפולס: שיקול נוסף הוא משך האות. תהליך המפולת עצמו מהיר ביותר – האלקטרונים מגיעים לחוט האנודה תוך סדר גודל של ננו-שניות. אולם, יוני הגז החיוביים הנותרים נסחבים לאיטם לקתודה (בתהליך שנמשך מיקרו-שניות ואף מילי-שניות, אך ניתן להפחית השפעתו ע"י מעגלי RC כפי שמוסבר בפרק האלקטרוניקה). את האות המהיר יוצר זרם האלקטרונים, והוא הקובע את חזית הפולס. חזיתות הפולס בגלאי גז פרופורציונלי הן בדרך-כלל בסדר גודל של עשרות ננו-שניות, והתארכות הפולס (זנב) יכולה להיות עד כמה מיקרו-שניות. עם מערך אלקטרוני מתאים (הטענה ושייפול), אפשר לקבל פולס רוחב של כ-0.5–1 מיקרו-שניה נוח למדידה. 2.3 חישובי קיצון וניתוח ביצועים כדי להבין את השפעת הפרמטרים על ביצועי הגלאי, נבחן כמה "תסריטי קצה" בהשוואה לתצורה הנומינלית: אלפא חלש (1 MeV) מול אלפא חזק (5–10 MeV): במקרה של 1 MeV, כפי שציינו, ייתכן שלא כל האנרגיה נבלעת (מסלול קצר). אפילו אם נניח שנבלעה, $N_0$ קטן פי $\sqrt{10}$ מזה של 10 MeV, כלומר אולי רק 1/10 מהיונים. לכן פולס של 1 MeV בהגבר נתון יהיה קטנטן – למשל 0.1–0.2 V אם 5 MeV נתן 1 V. דבר זה מדגיש את הצורך בהגבר גזי: ללא הגבר, 1 MeV אלפא היה נותן רק כ-30,000 אלקטרונים, שהם $4.8\times10^{-15}$ C, כלומר $\sim0.16$ mV על קבל 30 pF – איבוד מוחלט ברעש. ההגבר הופך 0.16 mV ל-160 mV, הבדל של שלושה סדרי גודל בקלות הזיהוי. מצד שני, 10 MeV אלפא (אם היה) בהגבר גבוה היה נותן פולס ענק, אולי 5–10 V, ועלול לגרום לרוויה או אפילו ניצוץ. לכן יש לאזן: במידה וטווח האנרגיות הצפוי שלנו הוא 1–5 MeV, ניתן לכוון את מתח העבודה כך שאלפא 5 MeV נותנת למשל 2 V, ואלפא 1 MeV תיתן ~0.4 V – שניהם בטווח המדידה התקין. אם לעומת זאת הטווח היה 1–10 MeV, היינו עשויים להציב מתח מעט נמוך יותר כדי שלא לרוות ב-10 MeV, במחיר ירידה מסוימת ברגישות ל-1 MeV. שינוי בלחץ או בטמפרטורה: אוויר כלוא במיכל יכול לשנות לחץ עם הטמפרטורה (חוק גז אידיאלי). עלייה בטמפרטורה או ירידה בלחץ האטמוספירי תקטין את $p$, מה שיוריד את $V_{\text{min}}$ ויקל על ההגבר (אך גם אומר שפחות התנגשויות – פוטנציאלית קצת פחות יונים ראשוניים). ירידת טמפרטורה תעלה את $p$ ותקשה על הגבר. כעיקרון, שינויים קטנים (כמה מעלות) אינם משמעותיים מאוד, אך אם עובדים בתנאי סביבה משתנים כדאי לקחת מרווח ביטחון במתח כך שגם בתנאי הקשים ביותר יש עדיין הגבר מספק. לחלופין, ניתן לפצות ידנית – למשל להגביר את מתח האספקה ביום קר כדי לשמור על אותה רמת פולסים. השפעת אי-אחידות השדה (מרווחי חוטים גדולים): אם החוטים רחוקים זה מזה, יש "כיסים" בין החוטים שבהם שדה החשמל חלש יחסית. אלקטרון ראשוני שנוצר באמצע בין שני חוטים יצטרך לנוע מרחק גדול יותר עד שיגיע לאזור שדה חזק. לאורך תנועה זו, אם השדה חלש, הוא עלול להתאחות (רקומבינציה עם יון חיובי) או להיקלט ע"י מולקולת חמצן ולהפוך ליון שלילי איטי. בכך האלקטרון "יאבד" ולא יגיע למפולת, או שיגיע מאוחר (יון שלילי נע הרבה יותר לאט). כתוצאה, אירועים שקורים רחוק מדי מחוט עשויים לתת פולסים חלשים מהצפוי, או לא להיספר כלל. זה מוביל לירידה ביעילות הגילוי ולאי-יציבות בספקטום (רוחב פסגות מוגדל). לכן, מרווח חוטים קטן הוא חיוני ליעילות: נרצה שכל נקודה בנפח הגלאי תהיה במרחק לא יותר מכמה מילימטרים מחוט אנודה כלשהו, כדי להבטיח שגם אלקטרונים מאירועי יינון בנקודה זו ינועו די מהר ויגיעו לאנודה לפני אובדן. בהמשך נחשב בדיוק את המרווח המיטבי בהתאם לגיאומטריה (סעיף 7). קוטר חוט קיצוני: אם היינו מגדילים מאוד את עובי חוטי האנודה (למשל לחצי מ"מ), השדה על פניהם קטן משמעותית (כי $a$ גדול, $\ln(b/a)$ קטן). כפי שראינו, מתח הסף הדרוש להגבר יעלה לאלפים רבים של וולטים – לא מעשי ולא בטוח. מצד שני, אם היינו מצליחים להשתמש בחוטים עוד יותר דקים מ-20 µm (יש שמנסים 10 µm ואף פחות), היינו מורידים עוד את המתח הדרוש אך החוטים היו שבירים מאוד וקשים לטיפול. בחירה של ~20 µm היא איפוא פשרה טובה: מספיק דק כדי ש-1–2 kV יספיקו להגבר, ומספיק חזק מכנית (פלדת טונגסטן מצופה זהב) כדי לא להיקרע במתח ובהרכבה. סוג גז שונה: אמנם הפרויקט מגדיר אוויר כלוא, אך למחשבה – אילו היינו משתמשים בארגון טהור, או הליום, או תערובת קרובה (P-10), היינו מקבלים ספים שונים. ארגון הוא גז "קל" ליינון (ערך $K$ נמוך יותר מאוויר) ולכן מתחי העבודה יכולים להיות נמוכים יותר בכ-20% אולי. הליום דורש מתח גבוה יותר (ערך $K$ נמוך אבל $\Delta V$ גדול כי אלקטרון מתקשה ליינן הליום ללא אנרגיה גבוהה), אך הליום כמעט לא לוכד אלקטרונים (אין O_2 בכלל) ולכן יכול לתת ספקטום חד. שילוב גז "מרפה" (כמו מתאן, פחמן דו-חמצני, פרופן וכו') משפר מאוד את יציבות ההפעלה: הוא מונע מן היונים החיוביים לעורר פליטת אור אולטרה-סגול שיכול לגרום לפעולת זיק ותופעות מיותרות nuclear-power.com . באוויר, התפקיד הזה משוחק חלקית ע"י $N_2$ (יש לו מצבי עירור רוטציוניים המשמשים כמרפה), אך $O_2$ פועל הפוך (לוכד אלקטרונים). לכן אנו מצפים שגלאי אוויר יהיה בעל מתח הצתה גבוה מעט וגמול הגבר מעט נמוך לעומת גלאי עם תערובת אופטימלית. למרות זאת, עבור חלקיקי אלפא חזקים הבעיה פחות חריפה – כמות היונים הראשונית גדולה מאוד כך שגם הגבר נמוך מספיק. ניתן לסכם כי הביצועים האידיאליים יושגו כאשר: המתח מכוון מעט מעל סף הגילוי, החוטים קרובים ודקים, והגלאי נשמר נקי ויבש למניעת פריצות ולקיחת אלקטרונים. כל חריגה מפרמטרים אלו – למשל מתח גבוה מדי – עלולה לגרום איבוד פרופורציונליות, ניצוצות, או קיצור חיי הגלאי, ואילו מתח נמוך מדי או מרווח גדול מדי יגרמו לאיבוד חלק מהאירועים (חוסר יעילות) ולרזולוציה גרועה. בפרקים הבאים נשתמש בתובנות אלו לתכנון המעשי של רכיבי הגלאי: המבנה המכני, אספקת המתח, מערך הגז, האלקטרוניקה, ועוד. 3. תכנון מכני של הגלאי (מידות, מרכיבים ואילוצים הנדסיים) התכנון המכני של גלאי ה-MWPC צריך לענות על מספר דרישות עיקריות: גיאומטריית האלקטרודות: לכלול מערך של חוטי אנודה דקים במרווח מתאים, עם אלקטרודות קתודה מוליכים סביבם (לוחות או רשתות) ליצירת שדה אחיד ככל הניתן. ממדים כלליים: אורך ורוחב הגלאי יקבעו את שטח הכיסוי (כמה גדול האזור הרגיש). עומק הגלאי (מרחק בין הקתודות) צריך להיות מספיק כדי להכיל את מסלולי האלפא, אך קטן מספיק כדי לא לבזבז נפח ולשמור על שדה גבוה. הוגדר שעומק של עד 2 ס"מ הוא היעד. מבנה מיכל אטום: מכיוון שהגז כלוא (לא זורם), גוף הגלאי צריך להיות אטום כדי לא לאבד גז ולא להכניס אוויר לח (אשר יכול לפגוע בתפקוד). כמו כן, יש לכלול חלון כניסה אם רוצים לזהות חלקיקים ממקור חיצוני. קשיחות מכנית והרכבה: המערכת צריכה לאפשר מתיחת החוטים ושמירתם במקום לאורך זמן, מבלי שיגעו בקתודה או זה בזה. החוטים צריכים להיות מבודדים מכנית וחשמלית מהמארז, פרט לנקודת חיבורן לחשמל. עמידות במתח גבוה ובטיחות: כל חלקי המבנה הקרובים לחוטים יהיו תחת מתח של עד 2 kV, ולכן צריך למנוע פריקות (edges חדים וכו'), להבטיח מרחקי בטיחות בין מוליכים, ולבחור חומרי מבנה מבודדים מתאימים. 3.1 מבנה אלקטרודות: חוטי אנודה וקתודות אנודות: נבחר חוטי טונגסטן מצופי זהב בקוטר $\sim20~\mu$m (0.020 מ"מ). חוטים אלו נפוצים מאוד בגלאי חלקיקים, שכן לטונגסטן חוזק מתיחה גבוה ונקודת התכה גבוהה (לא ניתך בניצוצות קטנים), וציפוי זהב מונע חמצון ומשפר מוליכות. את החוטים נמתח במסגרת כך שהם מקבילים זה לזה במישור אחד. אפשר למתוח אותם למשל אופקית (כיוון X) ובמרחקים שווים לאורך כיוון Y של המסגרת. קתודות: יש כמה אפשרויות לתכנן קתודות במבנה MWPC שטוח: זוג לוחות מתכת מקבילים מעל ומתחת למישור החוטים, המרוחקים כ-10 מ"מ ממישור החוטים (זה ייתן עומק כולל 20 מ"מ). הלוחות יכולים להיות יריעות אלומיניום דקות, נחושת מודבקת לפרספקס, או כל משטח מוליך חלק. היתרון: שדה אחיד בצדדים ושמירה על פשטות. חסרון: אין רגישות מיקום בציר Z (רק החוטים נותנים מיקום בכיוון X, אולי נוסיף מערך חוטים נוסף בכיוון Y בין הקתודות אם רוצים גם Y). כמו כן, לוחות אטומים יפריעו אולי לכניסת אלפא אם מגיעים מצדדים. שתי רשתות (grid) של חוטים עבים יותר או מסגרת מחוררת, שיכולות לתפקד כקתודות "שקופות" יחסית. למשל, חוטי מתכת של 0.5 מ"מ מוצבים בקצוות התא במרווחים צפופים (1–2 מ"מ) כך שהן יוצרות משטח רשת. שדה החשמל יהיה כמעט אחיד אך הרשת מאפשרת לקרינה להיכנס ויש גמישות להסתכלות פנימה. שילוב: קתודה תחתונה (או עליונה) יכולה להיות גם שכבת חומר המשמשת כחלון כניסה. למשל, אם נשתמש ביריעת Mylar דקה מצופה זהב כחלון עליון, היא גם תשמש כקתודה אם נחברה להארקה. נניח בתכנון שלנו שנבחר בפשטות: שני לוחות מתכתיים מקבילים כקתודות, אחד מעל ואחד מתחת למערך החוטים. הלוחות יהיו עשויים אלומיניום בעובי 1–2 מ"מ (כדי שיהיו קשיחים ולא יתעקמו תחת מתיחת ברגים). ניתן גם להשתמש בכרטיס פיברגלס מצופה נחושת (PCB) כדי ליצור קתודה – יש לנו אז גמישות להוסיף עליו דוגמאות (למשל פתח או חלון). הלוח העליון נקדח במרכזו חלון עגול בקוטר ~5 ס"מ, שמכוסה בחלון כניסה שקוף לאלפא (ראו דיון בהמשך). סביב החלון, יתרת הלוח משמשת קתודה. הלוח התחתון יכול להיות שלם, או גם עם חלון (אם רוצים אפשרות כניסת אלפא מתחת). כרגע נניח שהמקור שלנו יונח מלמעלה, לכן רק למעלה צריך חלון. הלוחות יחוברו לדפנות המארז באמצעות מבודדים, ויקבלו פוטנציאל חשמלי נפרד (למשל אדמה – 0 V, בעוד האנודות יהיו במתח גבוה חיובי). מערך החוטים: אורך החוטים יקבע על-פי ממד התא. נניח שאנחנו מעצבים תא מרובע 10×10 ס"מ של שטח פעיל. החוטים ימתחו לרוחב 10 ס"מ (אורך כל חוט קצת מעל 10 ס"מ כדי לקבע בקצוות). מרווח החוטים – שאלה מרכזית: כנזכר לעיל (סעיף 2.3), כדי לא לאבד אלקטרונים עלינו לצופף אותם. אם נבחר מרווח גדול מדי, נפגע ביעילות. ערך סביר לנסות: מרווח 4–5 מ"מ בין החוטים. בפער כזה, מחושב שהמרחק המרבי מאמצע בין-שני-חוטים עד חוט הוא 2–2.5 מ"מ, וזה צריך להיות בסדר – אלקטרונים שנוצרים באמצע ינועו 2 מ"מ עד לחוט, בתקווה שזה מספיק קצר (זמן תעופה אולי 10 ns) לפני שילכדו. ייתכן שאף כדאי 3 מ"מ, אך אז יש הרבה חוטים (לכל 10 ס"מ יהיו ~33 חוטים). 5 מ"מ יתנו ~20 חוטים על 10 ס"מ, שזה מספר נוח יותר של ערוצים. נמשיך עם 5 מ"מ כרפרנס, ונראה בהמשך אם כדאי לשנות. שיקול נוסף: אם נרצה גם רזולוציה מרחבית (לדעת את מיקום הפגיעה על ציר X), אז מרווח 5 מ"מ יתן רזולוציה של 5 מ"מ בערך – תלוי אם החלקיק זוהה רק בחוט אחד או אולי בשני חוטים סמוכים (אפשר אינטרפולציה אם שניים נדלקו). ככל שמצופפים, רזולוציה משתפרת אך מורכבות הערוצים גדלה. החוטים יחוברו כולם חשמלית יחד או בנפרד? יש שתי אפשרויות לתפעול MWPC: חיבור משותף (common anode): כל חוטי האנודה מקושרים אחד לשני חשמלית (למשל מולחמים למסגרת מתכת משותפת). כך בעצם הגלאי מתפקד כגלאי יחיד גדול – כל פגיעה תגרום פולס באחד או כמה חוטים, אבל הם מחוברים, אז מה שרואים זה פולס כולל. שיטה זו פשוטה (ערוץ אלקטרוני יחיד), טובה אם רוצים רק למנות חלקיקים וכלל לא אכפת איפה פגעו. החיסרון: מאבדים את מידע המיקום, וגרוע מכך – עלולות להיווצר השפעות עומס (pulses בוזמנית על כמה חוטים יתערבבו). חיבור נפרד (individual wires): כל חוט מחובר לערוץ מגבר נפרד. כך מקבלים "רב-ערוצי" ויודעים בדיוק איזה חוט קלט את הפולס. זה אופטימלי למידע ואיכות הגלאי, אך מצריך הרבה אלקטרוניקה (מגבר לכל חוט, סה"כ 20 בערך). בפרויקט מעבדה בינוני זה אפשרי, 20 ערוצים אינם סוף העולם – ניתן להשתמש בכרטיסוני מגבר ומשפעני זרם וכדומה. חיבור קבוצות (segments): פשרה היא לחלק את החוטים לקבוצות (למשל 4 קבוצות של 5 חוטים על-כל 10 ס"מ). כך קצת מידע מיקום נשמר (איזה קבוצה), אבל מורידים כמות ערוצים. בפרויקט שלנו אולי לא חיוני. בהנחה שהמטרה העיקרית היא לספור גלאי אלפא ולהדגים עקרון, אפשר לעשות חיבור משותף. אך מכיוון שרוצים "עבודת תזה מקיפה", נניח שנאמץ חיבור נפרד (זה מעניין יותר לכתוב עליו). לכן החוטים יהיו מבודדים אחד מהשני בחיבורים שלהם. כיצד מעשית? כל חוט יותקן כך ששני קצותיו נמתחים לנקודות עיגון. בצד אחד אפשר לחבר אותו לחשמל (דרך נגדי הגנה וכו', ניגע בזה בחשמל), ובצד שני ניתן לחבר אותו דרך נגד לסיגנל או למתח (יש טכניקות שונות: למשל אחד הקצוות מחובר למתח HV דרך נגד גדול, והשני דרך קבל למגבר – כך DC כולם ב-HV, AC יוצא למגברים). נפרט זאת בפרק האלקטרוניקה (איך לאסוף את האות מכל חוט). מבחינה מכנית, זה אומר שצריך מערכת feedthroughs: מעבר מבודד לקיר המארז עבור כל חוט כדי להוציא את האות שלו. אפשר לתכנן זאת: להשתמש בלוח מעגל מודפס (PCB) משני צידי התא – בכל צד לרתום קצות החוטים. ה-PCB יכול לכלול פדים להלחמת החוטים, וכן נגדים/קבלים במידת הצורך, ומחבר (connector) כדי לחבר את הערוץ לכבל אלקטרוני. בצורה זו מקבלים מבנה מדויק (PCB מבטיח מרווח אחיד), חיבור חשמלי לכל חוט, ואפשרות ליישם סיכום או הפרדה בקלות. לחלופין, ניתן פשוט למתוח את החוטים על גבי מסגרת ולחבר ידנית חוטי יציאה – פחות אלגנטי, מסובך לכוון מרווחים. כיוון שאנו מעצבים "תזה" ולא בהכרח בונים בפועל, נבחר בדרך המסודרת: נכין שני לוחות PCB עם חריצים, עליהם מולחמים קצות חוטי האנודה. הלוחות הללו גם ישמשו כדפנות הצדדיות של התא (אולי ייסגרו את ימין ושמאל). ה-PCB העליון והתחתון יכולים לשמש כקתודות אם מצופים – אבל הקתודות העיקריות הן הלוחות הגדולים. למעשה, כדי שלא תהיה הפרעה, נעשה חתך בקטעי PCB ליד החוטים כדי שלא תהיה מוליכות קרובה מדי. יש פרטים קטנים, אך לא נרד אליהם כאן. סיכום בחירת מידות: שטח פעיל: 10 ס"מ x 10 ס"מ (יכול להיות כל גודל, 10 ס"מ מספיק להדגים). עומק: 2 ס"מ (קתודה-לקתודה). אנודות באמצע (מרחק 1 ס"מ מכל צד). אנודות: חוטים 20 µm, מרווח 5 מ"מ, ~20 חוטים. מתוחים בכיוון X. קתודות: לוחות אלומיניום, עם חלון כניסה עליון בקוטר 5 ס"מ (מכוסה יריעת מיילר דקה). מסגרת ומארז: אפשר לתכנן קופסה מלבנית, חומר אקרילי (פרספקס) או PVC מוקצף, עם אטמי O-ring למנוע כניסת אוויר. הברגות בפינות להצמדת הלוחות, ותושבות ל-PCB הצדדיים. כניסת כבלים: יהיה כבל HV עבור האנודות, וכמה כבלי אות (BNC או מחברים מרובים) עבור האותות. יש להקפיד שהמעבר אטום (אפשר SMA אטום עם O-ring, או מחברי קרמיקה ייעודיים למתח גבוה). אילוצים הנדסיים: מתיחת חוטים: החוטים צריכים להיות מתוחים היטב כדי שלא יתרופפו ויגעו בדופן או יתקרבו זה לזה. מתיחה מושגת ע"י הידוק קצה החוט בעוגן אחרי מתיחה מכנית (קפיץ או משקולת זמנית). אפשרות נוספת: לשים קפיצים זעירים בכל קצה שימשיכו למתוח. קוויברציה ורעידות: חוט דק יכול להתנדנד (ויברציה) אם יש זעזוע. זה מסוכן כי אם חוט יתקרב מדי לקתודה – ניצוץ. לכן במסגרת התכנון נוסיף distancers או נצמצם את אורך החופשי. ייתכן שנמקם מרווחי פלסטיק בין חוטים באמצע הדרך כדי לייצב (אם כי זה עלול להפריע לשדה). ברוב המקרים לא נוגעים, אלא סומכים על מתיחה טובה. חומרים מבודדים: המארז, תושבות הברגה, וכל חלק קרוב למוליכים צריך להיות מחומר בעל חוזק דיאלקטרי גבוה. אפשר פרספקס (PMMA) שהוא מבודד די טוב ל-2 kV על ס"מ. להימנע משימוש במתכות או מוליכים ליד נקודות HV ללא בידוד. הברגים שקובעים קתודה צריכים להיות שקועים או רחוקים מהחוטים. זוויות חדות במוליכים (כמו פינות של הלוח הקתודה) כדאי לעגל מעט, כי במקום חד מתרכז שדה ועלול להיווצר corona. דליפות גז: כגלאי אוויר כלוא, אנו למעשה יכולים לבנות לא אטום הרמטית (מכיוון שמדובר סה"כ באוויר, לא נורא אם קצת זולג פנימה). אך עדיף לסגור כמה שניתן למנוע כניסת לחות ואבק. אטמים מגומי סיליקון יכולים לשמש בממשקי המארז. חיבורי המחברים (BNC, SHV) יש לאטום עם אפוקסי או להשתמש בסוג אטום. 3.2 חלון כניסה לחלקיקי אלפא חלון הכניסה הוא חלק קריטי אם המקור מחוץ לגלאי. חלקיקי אלפא, כאמור, נבלמים בקלות – אפילו כמה עשרות מיקרון של חומר עלולים לבלוע 5 MeV אלפא. לכן החלון חייב להיות דק ככל האפשר. פתרונות נפוצים: יריעת Mylar (פוליאסטר) בעובי 1–2 µm, מצופה שכבה דקה של אלומיניום (כמה עשרות ננומטר) בשביל הולכה חשמלית. יריעה כזו מאפשרת מעבר אחוז גדול מהאלפא (איבוד אנרגיה של אולי 0.5–1 MeV בכניסה), ומחזיקה מעמד בהפרש לחצים קטן. צורה אחרת: חלון בקוטר קטן (נגיד 1 ס"מ) עשוי סרט Kapton דק. לחלופין, אם אפשר שהמקור יהיה בתוך הגלאי, אז אפשר לוותר על חלון. לדוגמה, קפסולת ^241^Am אפשר לפתוח ולהציב פנימה (אם כי זה מפזר חומר רדיואקטיבי, לא הכי טוב) או לשים את כל המקור ממש צמוד על פתח פתוח ולהצמיד כדי שלא יברח אוויר. כאן נניח שנתכנן חלון: בקתודה העליונה נפתח חור גדול (5 ס"מ) ומדביקים עליו מבפנים יריעת Mylar 2 µm מצופה אלומיניום. ציפוי האלומיניום יהווה חלק מהקתודה העליונה (במקום האלומיניום שהיה שם). נוודא להדק סביב בחומר אפוקסי או אטם כדי לא לזלול אוויר. חלון כה דק בלחץ atm? – לעיתים תומכים אותו עם רשת תיל דקה (mesh) כלשהי כדי שלא יקרע בלחץ, במיוחד אם חיצון-פנימי יש הפרשי לחץ. במקרה שלנו, פנים הגלאי הוא 1 atm ואותו לחץ בחוץ, אז בהפעלה אין הפרש (אלא אם כן טמפרטורה שונה). אך כשהרכבנו, אם סגרנו את התא ביום חם וקררנו, יווצר תת-לחץ פנימי שעלול למצוץ את החלון פנימה. ובכיוון הפוך – ביום חם הלחץ בפנים יעלה. כדי למנוע קריעת חלון, רבים מתקינים פורק לחץ: שסתום חד-כיווני קטן שמאפשר כניסת אוויר אם הלחץ הפנים יורד יותר מדי, או פתיחה ידנית לשחרור אם להיפך. נניח שנוסיף שסתום קטן. בחלון 5 ס"מ, Mylar 2 µm יכול לעמוד בכמה עשרות מיליבר הפרש בלי להיקרע. 3.3 מכלול הרכבה נתאר את סדר ההרכבה: מכינים את מסגרת המארז (למשל תיבה מלבנית ללא דפנות עליונות ותחתונות). במסגרת הזאת משחילים את לוחות ה-PCB הצדדיים עם חריצים לחוטים. מהדקים אותם עם ברגים. מותחים את חוטי האנודה: קושרים צד אחד לכל חור/פד PCB, משחילים לצד השני, מושכים עם פינצטה/מלקחיים למתח הרצוי (אפשר למדוד ע"י תדר תהודה – אם פורטים קלות על החוט, תדירות תנודה מעידה על מתיחה). מלחימים או מקבעים בקצה השני. חוזרים 20 פעמים לכל החוטים. (פעולה עדינה, רצוי תחת מיקרוסקופ). מחברים רכיבים פסיביים נחוצים על ה-PCB (לדוגמה נגדי עומס לכל חוט). מחברים את מחבר ה-HV: קצה כל החוטים כנראה ילך יחד ל-HV דרך נגדים (אם common HV) – יכולים לעשות את זה על ה-PCB. סוגרים את החלק התחתון: שמים את לוח הקתודה התחתון (אלומיניום) ומבריגים/מדביקים למקום עם אטם. מכינים את הלוח העליון: מדביקים חלון מיילר עליו ומוודאים מולטי-מטר שהוא יוצר מגע (האלומיניום שעל המיילר יוצר מגע עם שאר הלוח). סוגרים את הלוח העליון עם אטם, מבריגים. מבצעים בדיקת נזילה (אפשר לשים על ואקום עדין, לראות אם שומר, או עם גלאי הליום אם יש). אם מעט דולף לא נורא – זה אוויר. מחברים את הכבלים: כניסת HV (מחבר SHV לדופן, מוליך ל-PCB), כבלי אות (אם נפרד – 20 כבלים, אפשר להוציא עם סרט נחושת גמיש ל-20 כניסות בפתח). בדיקת חשמל: עם מולטי-מטר מכוונים שאין קצר בין HV לקתודה (התנגדות גבוהה – רק דרך נגדים), שכל חוט מבודד מדופן. ניקוי פנימי: לפני הסגירה הסופית טוב לנקות מבפנים עם אוויר יבש או חנקן כדי לסלק אבק. הגלאי מוכן לחיבור למערכת אלקטרונית. ייצוב, כיול ופיקוח: לעיתים מתקינים בתוך הגלאי חוט או טבעת "גריד" בין האנודה לקתודה כדי לקבל שדה אחיד יותר. כאן לא חובה. עוד דבר: אפשר לשים נורת ניאון זעירה או דיודת זנר בין האנודה לקתודה כ"הגנת מתח" – אם המתח עולה מעל ערך מסוים (ניצוץ), הנורה תוליך ותגן על החוטים. זה טריק בטיחות. יושם אם צריך. סיכום איורים טכניים: באיור 4 להלן מוצג שרטוט סכמטי של חתך הגלאי המוצע, המדגים את סידור החוטים, הקתודות והמארז. איור 4: מבנה סכמטי מוצע של גלאי MWPC מבוסס-אוויר. חוטי אנודה (מסומנים W באיור, בצבע שחור) נמתחים במרכז בין שני לוחות קתודה (P, בצבע כחול). חלקיק אלפא (מסלול מסומן T באדום) נכנס דרך חלון דק בחלק העליון, מיינן את הגז (הנקודות האדומות המפוזרות הן יוני גז), והאלקטרונים הנוצרים נאספים לחוטי האנודה הסמוכים. כל חוט מחובר למגבר (A בשוליים) הממיר את מטען המפולת לאות חשמלי (צורת הפולסים מוצגת מימין). שים לב שהשדה החשמלי בין האנודה לקתודה אינו מוצג במלואו; בקרבת כל חוט יש שדה רדיאלי חזק (משורטט באופן איכותני בקווים מתוחים סביב החוטים). תרשים זה מדגים את העיקרון אך אינו בקנה מידה. en.wikipedia.org en.wikipedia.org התכנון המכני כפי שתואר משלב את כל האלמנטים: הוא מספק מבנה איתן לחוטים, נפח גז של 2 ס"מ, אפשרות לחלון, ובידוד מתאים למתח גבוה. כעת, משהשלמנו את הפן המכני, נעבור לדון במתח הגבוה הנדרש להפעלת הגלאי ובשיקולי בטיחות. 4. מתח גבוה: טווח 1000–2000 וולט, השפעות ובטיחות טווח המתח ואופן היישום: כפי שנידון, הגלאי יתופעל במתח גבוה בתחום של בערך 1000 עד 2000 V. האנודות (החוטים) יהיו לרוב בפוטנציאל הגבוה, בעוד הקתודות (הלוחות והחלון) יהיו באפס (הארקה). ניתן גם לעשות להפך (אנודות באפס, קתודות במתח שלילי), אך מקובל מבחינת אלקטרוניקה שהאנודה בחוט דק היא חיובית – כך האלקטרונים מואצים אליה. בנוסף, זה מפחית רעש כי האלקטרוניקה המקבלת יכולה להיות קרובה לאדמה. בפועל נחבר את כל חוטי האנודה דרך נגדי שרשרת (כמה מגהאום כל אחד) למקור HV חיובי של נניח +1500 V. אספקת המתח הגבוה: נשתמש בספק כוח מתח גבוה (High Voltage Power Supply) מסוג שאפשר לקבוע עד 2000 V, בזרם יציאה מקסימלי קטן (כמה מיקרואמפרים עד מאיות מיליאמפר). זרם הגלאי במצב עבודה רגיל יהיה זניח – כל אלפא יוצרת פולס, ובממוצע אולי אלף פולסים לשניה לכל היותר (תלוי בקצב המקור). בהערכה גסה, אם כל פולס מעביר $10^8$ אלקטרונים (מטען $1.6\times10^{-11}$ C), ובקצב 1000 אירועים/שניה, הזרם הוא $1.6\times10^{-8}$ C/s = $1.6\times10^{-8}$ A (0.016 µA). זה ממש זניח. אפילו אם יהיו כמה רקעים, סה"כ <1 µA. לכן דרישות הספק נמוכות. חשוב שהספק יהיה יציב ומדויק, עם אפשרות שינוי מתח עדין (נרצה אולי לכייל 1200, 1300... לראות איך זה משפיע). התפלגות המתח בגלאי: כדי להגן על החוטים וכדי לקבוע היטב את שדה החשמל, בדרך-כלל מכניסים חלוקת מתח (divider). לדוגמה, אפשר לשים נגד סדרתי גדול (נניח 50 MΩ) בין הספק HV לכלל האנודות, כך שבמצב ניצוץ הזרם מוגבל. בנוסף אפשר לשים נגד לכל חוט (לדוגמה 5 MΩ בשרשרת לכל חוט). כך אם יש פריצה מקומית בחוט אחד, הנגד יפיל את רוב המתח ולא יאפשר זרם בלתי מוגבל. השיטה גם דואגת שכל חוט יושב במתח כמעט שווה כל עוד הזרמים זניחים. כאמור, ייתכן שנעצב את PCB האנודה כך שבכל חוט יש נגד טורי לערוץ ה-HV, ולקצה השני קבל למגבר. זה נקרא "Resistive isolation" ומונע זליגה של פולס מחוט אחד למשנהו. בטיחות חשמלית בגלאי: 1000–2000 וולט הם מתחים גבוהים שיכולים בהחלט לסכן חיים אם נוגעים בהם ישירות. על כן, כמה אמצעי בטיחות: כל מוליכי ה-HV יהיו מבודדים ומכוסים. למשל, חוט HV שמגיע לגלאי יהיה כבל קואקס עם בידוד עבה (כבל SHV סטנדרטי). נקודת הכניסה לגלאי – מחבר SHV שיש לו בידוד ואף מפתח מכני שלא יתחבר לכבל BNC רגיל בטעות. בתוך הגלאי, החיבור ל-HV והפס המוליך לשרשרת הנגדים יהיה מצופה בשכבת אפוקסי/סיליקון למניעת מגע בשום חלק אחר. הגלאי עצמו יהיה בדרך-כלל בתוך מארז סגור. כדי לבצע כיוונים או החלפת מקור, מכבים קודם את ה-HV ומפרקים רק כשהכל פרוק ופרוק חשמלית. נהוג לכלול במערכת HV גם נגד פריקה (bleeder) שמוודא שכאשר מכבים, המטען במוליכים מתאפס תוך כמה שניות. רצוי לשים נורה או חיווי כש-HV דולק, ואינטרלוק (מפסק בטיחות) שיכבה HV אם פותחים את מארז הגלאי. בנוסף לבטיחות האדם, ישנה בטיחות הגלאי עצמו: ניצוץ פנימי עלול להרוס חוט (לצרוב אותו או להתיך בנקודה דקה). לכן ההגנות כמו נגד טורי, או אפילו ניצול עקרון "קו לא משורשר" (un-terminated line) שמשאיר את האנודה צפה קצת כך שאין פריקה מלאה, אפשריים. חלק מזה כבר בטבע – הזרם מאוד מוגבל כי רק המטען בגלאי זמין. יש גם לשמור שהגלאי לא יהיה בקרבת חומרים דליקים כשהוא בפעולה – ניצוץ קטן בתוך אוויר יכול דרך חלון אולי להצית משהו בחוץ. אמנם הסיכוי קטן, אבל בכל זאת HV נוהגים להרחיק מניירת וכו'. השפעות המתח על הגלאי: אם המתח גבוה מדי: נקבל תחילה זרם רקע מוגבר (Dark current) – בגלל קרינה קוסמית, רדיואקטיביות רקע ואפילו פליטות תרמיות, כל אלה עם הגבר גבוה יתנו פולסים אקראיים. נראה ספירה גם בלי מקור. בנוסף, נראה שהרעש האלקטרוני גדל (למשל זרם זליגה על פני המבודד יוצר אותות). הגבול העליון יהיה כשמתחילות פריקות זעירות: קורה כששדה מתקרב לערך פריצה במקומות מסוימים (למשל ליד פגם בחוט או גרגר אבק). זה יתבטא בפולסי ניצוץ גדולים (ממש כמו GM) ואי-יציבות (count rate מזנק, הציוד אולי saturates). אסור להגיע לזה – אם רואים "זנב ארוך" בפולסים או קפיצות לא מוסברות, כנראה על הסף. צריך להפחית מתח. אם המתח נמוך מדי: לא כל האירועים יירשמו. יכול להיות סף שבו רק אלפא של 5 MeV גלים, אבל 2–3 MeV כבר לא עוברים את סף האלקטרוניקה (או אפילו לא מגיעים למפולת). הדבר יוצר איבוד יעילות סלקטיבי. גרף הספירה כתלות במתח (Plateau curve, ראו דיון בסעיף 7) יראה עלייה בהתחלה, ואז יתיישר – סימן שהגענו למתח שבו כבר כמעט כל היונים הראשונים נותנים פולס מעל סף הגילוי mdpi.com mdpi.com . אזור מתוח זה הוא הרצוי – מה שנקרא "פלטו" (platau) – מתח שאפשר לשנות קצת בלי לאבד מיידית ספירה. מתח משתנה: רצוי ספק HV יציב. ריפל או רעשים בספק (כמו 50 Hz מהמעט) יכולים להכניס רעש מדידה (פולסים גדלים וקטנים מעט). לכן ספקי HV איכותיים משתמשים בריפל מיליוולטים על אלפי וולט. יש לכלול מעגל יצוב וכנדרש להאריק את כל המערכת היטב. בדיקות ובטיחות לפני הפעלה: לפני שמחברים מתח גבוה לגלאי בפעם הראשונה, עושים בדיקת עמידות: מעלים את המתח לאט (למשל 100 V כל פעם), ומוודאים שאין פריצה. אפשר לבצע זאת בחושך מוחלט כדי לראות אם יש נצנוץ קורונה איפשהו. אם כן – לעצור ולתקן (איטום אזור או בידוד). לעיתים אפילו בלי לראות אור אפשר לשמוע תקתוק או זמזום אם פריקה. בסביבת 2000 V, אם הגלאי נקי ואטום, לא אמורים להיות פריקות. בנוסף, לעולם לא משאירים HV פועל ללא השגחה מתאימה, ותמיד פורקים את המתח לפני פתיחה. שילוט "High Voltage" צריך להיות על המארז כדי להזהיר. לסיכום חלק זה: מתח ההפעלה 1–2 kV הוא תנאי הכרחי לפעולת MWPC באוויר, והוא דורש תשומת לב ומערכי בטיחות. המתח ישפיע ישירות על מקדם ההגבר, ולכן גם על הרזולוציה והיעילות: מתח גבוה יותר – יותר אות (טוב) אבל יותר רעש וסיכון (רע); מתח נמוך – בטוח ושקט אך פחות רגישות. נצטרך למצוא את נקודת העבודה האופטימלית באמצעות עקומת הפלס (ראו בהמשך, סעיף 7). 5. מערכת הגז: אוויר כלוא לעומת מערכות גז פתוחות הבחירה בבניית הגלאי עם אוויר כלוא (Sealed Air) לעומת מערכת גז פתוחה (זרימת גז קבועה) משפיעה על תפעול ותחזוקת הגלאי: אוויר כלוא (Sealed Chamber): במקרה זה, אנו ממלאים את נפח הגלאי פעם אחת (בזמן ההרכבה) באוויר ברמת ניקיון ולחות ידועה, ואוטמים אותו. הגלאי יכול לפעול כך זמן רב ללא תלות באספקת גז חיצונית. היתרונות: פשטות: אין צורך בבקבוקי גז, בוסתים, צינורות ומסתמים. הגלאי נייד וקל להפעלה – רק מחברים חשמל והוא מוכן. אחידות: בהרצה אחת, תנאי הגז קבועים (אותו לחץ, אותו הרכב) כל עוד האטם מחזיק. כך הכיול שנעשה נשמר. אין צריכת גז: חוסך בעלויות תפעול (אם כי אוויר זול מאוד בכל מקרה). אך יש גם חסרונות שצריך לשים לב אליהם: זיהום הדרגתי: עם הזמן, מזהמים יכולים להשתחרר מפנים המארז (outgassing). למשל, דבק אפוקסי, פלסטיק, חומרי הלחמה – פולטים אדים קלים שיצטברו. הללו יכולים לספוח אלקטרונים (אם הם מכילים חמצן או מולקולות אלקטרושליליות) או לשנות את לחץ הגז. תוצרי קרינה (למשל אלפא יכולה לגרום לפירוק מולקולות וליצירת רדיקלים או אוזון) גם כן נותרים בפנים. לאורך זמן יתכן שתגובת הגלאי תשתנה (ירידה בהגבר, הופעת פריקות פתאומיות). זו "הזדקנות" ידועה בגלאי גז. לחץ וטמפרטורה: בהיעדר ויסות, שינויים סביבתיים ישפיעו מייד. למשל, יום חם יעלה את הלחץ הפנימי (אם לא אפשרנו שחרור). לחץ גבוה אומר מס' התנגשויות יותר – אולי טיפה שינוי $M$. כנ"ל לחות: אם האטם לא מושלם, לחות יכולה לחלחל עם הזמן ולהצטבר בפנים, בעיקר אם טמפרטורה יורדת, המים יתעבו. לחות באוויר רע לגלאי – מים הם מולקולה פולרית ועלולים לגרום לגלאי לפרוץ מוקדם יותר, או לפחות להעלות זרם רקע. אין רענון: בגלאי עם זרימת גז, בכל אירוע הגז המיונן מתחלף די מהר (תוך שניות, תלוי קצב זרימה). כאן, היונים החיוביים שנוצרים (למשל $N_2^+$, $O_2^+$) בסוף יתנגשו בקתודה, יקלטו אלקטרון ויהפכו חזרה לניטרל, משחררים אנרגיה (פוטון UV אולי). הפוטון יכול ליינן מולקולה אחרת, וזה יוצר רקע (אפילו בלי קרינה). בגז זורם היו סוחפים זאת. בגלאי אטום, הרקע יכול בהדרגה לעלות. בפועל באלפא זה כנראה לא דרמטי – קצב הרענון הטבעי (דיפוזיה) אולי מספיק, והריכוז קטן. אבל לדוגמה, בגלאי נייטרונים מצופים בורון, מאוד חשוב זרימה כדי לפנות עודף $BF_3$ וכו'. מערכת גז פתוחה (Flowing Gas): זה המצב השכיח בגלאי חלקיקים גדולים – מחברים צינור גז, וזרם איטי של תערובת שומר על הרכב קבוע ומפנה זיהומים. היתרונות: גז ניתן להתאים אידאלית (P-10 למשל, או תערובת עם iso-Butane וכו'). הרענון המתמיד מונע הזדקנות: אדי דבק אם נפלטו – נשטפים, תוצרים – יוצאים. אפשר לווסת לחץ – להפעיל אולי ב-0.9 atm להוריד $V_{\min}$ וכו'. אם רוצים לכבות את הגלאי, אפשר גם למלא בגז נייטרלי וכו'. החסרונות: מערכת מסורבלת: צריך בלוני גז (לא תמיד בנמצא, דורש גם בטיחות למשל בלון $CH_4$ זה דליק). עלות ותפעול: צריך תמיד לוודא שיש זרימה, ששסתום פתוח, שלא נגמר בלון באמצע מדידה כי אז היחס משתנה. תנודה: זרימה לפעמים גורמת רטט או רעש מכני. לחות: אם יש טעות יכול להיכנס לחות מהסביבה דרך המערכת – אבל זה גם בגלאי סגור אם דולף. בפרויקט מעבדה לרוב יעדיפו אוויר סגור כי זה קל. אם רוצים לשפר, אפשר "לרמות" ולעשות מערכת חצי-אטומה: לסגור את הגלאי אך להשאיר פתח מילוי קטן שבאמצעות מזרק או בלון ניתן להחליף אוויר מדי פעם. למשל, אפשר לשאוב ואקום חלש ולמלא מחדש באוויר יבש. במקרה שלנו – אוויר כלוא: נצטרך להקפיד על: בנייה מחומרים עם פליטה נמוכה (לדוגמה: שימוש באפוקסי באיכות גבוהה נטול ממסים, ניקוי כל חלק עם אלכוהול לפני הסגירה להסיר שומנים). הכנסת מסנני לחות: אפשר לשים בתוך התא חבילת silica gel קטנה שתספח לחות. זה מקובל באריזת גלאים. שמירת התא סגור הרמטית ככל האפשר, מלבד שסתום בטחון. מילוי ראשוני באוויר יבש: אם למשל בונים ביום לח, אפשר לחבר צינור של גז יבש (חנקן או אוויר יבש מבקבוק) ולשטוף את התא לפני הסגירה. אחר כך לסגור תוך זרימה (שיישאר אוויר יבש בפנים). זה נותן התחלה מעולה עם 0% לחות, ואז גם אם קצת דולף לחות, יקח זמן להגיע לרמות בעייתיות. תפעול ארוך טווח: ייתכן שאחרי זמן מה, נצטרך לפתוח ולנקות אוויר, במיוחד אם שמים לב שהמתח ניתוק (breakdown) יורד – סימן שיש זיהום. או אם היעילות יורדת (יתכן $O_2$ נכנס ותופס אלקטרונים). אפשר לתקן זאת ע"י פתיחה, איוורור וייבוש מחדש, או פשוט פתיחת שסתום והזרמת אוויר טרי לכמה דקות, ואז אטימה. חשוב גם לזכור שאם הגלאי יעבוד מאוד חזק (קצב גבוה), נוצר הרבה $O_3$ (אוזון) ו-$NO_x$ שיכולים להיות בעייתיים, אבל בקצב אלפא קטן (מקור ^241^Am טיפוסי - 1 µCi - פולט כ-$3.5\times10^4$ אלפא בשנייה לכל הכיוונים, אז לתא שטח קטן אולי 1000 לשניה נכנסים) – לא כזה נורא. אוזון, עם הזמן, יכול לתקוף פלסטיק או דבק. זה עוד מניע להחלפת גז תקופתית. השוואה כמותית: אם היינו שמים P-10 זורם לעומת אוויר סטטי: P-10 דורש אולי 700 V למפולת לעומת אוויר 1000 V. P-10 נותן כנראה ספקטום אלפא חד יותר (כי אין O_2). באוויר אולי נראה פסגות קצת מרוחות (fluctuations כי חלק אלקטרונים נלכדים). יציבות: P-10 לאורך שנים נשאר טוב עם זרימה, אוויר סטטי אולי אחרי כמה שבועות רואים ירידה. אבל אוויר זמין, לא דליק, לא רעיל, ובלי סיבוכים. לסיכום חלק זה: מערכת האוויר הכלוא היא פתרון פשוט ונוח עבור סביבת ניסוי במעבדה, במיוחד לפרקי זמן קצרים ובינוניים. היא דורשת תשומת לב לאטימה ולניקיון הגז, אך חוסכת התעסקות בצנרת גז. לשימוש ממושך או מקצועי, אולי היה עדיף תערובת גז עם זרימה איטית, אך במסגרת פרויקט סטודנטיאלי – אוויר כלוא נראה מספק. 6. מערך אלקטרוני מלא: מגברי מטען, תנאי טריגר ומערכת DAQ הגלאי עצמו – חוטים, מתח גבוה וגז – מהווים רק חצי מהמערכת. החצי השני הוא האלקטרוניקה לקריאת האותות, עיבודם והקלטתם (DAQ – Data Acquisition). כיוון שמדובר בגלאי פרופורציונלי, האותות הם פולסים אנלוגיים מהירים. המערכת האלקטרונית צריכה לבצע כמה פעולות מפתח: איסוף המטען והמרתו למתח – באמצעות מגבר מטען (Charge Sensitive Preamplifier) או הגברה שוות-ערך. עיצוב האות (Shaping) – כדי להפוך את הפולס לחד וקצר מספיק למדידה, לסנן רעשים ולשפר יחס אות-רעש. סף והטריגר (Discrimination) – זיהוי פולסים מעל סף מוגדר, והמרתם לאותות דיגיטליים לספירה או לתזמון. מדידה כמותית (ADC) – אם רוצים למדוד את גודל הפולס (למטרות ספקטראליות), יש למצות ערך משרעת או מטען ולמיר למספר, באמצעות ממיר אנלוגי-לדיגיטל (ADC) או סוקר ערוצים (MCA – Multichannel Analyzer). מערכת רכישה – שיטות לשמור את הנתונים: למשל מונה קצב (Scaler) אם רק סופרים קצב, או מחשב שמקבל את הארועים (למשל בכרטיס נחשב או בארדו) כדי להפיק נתונים. כעת נפרט את הרכיבים: 6.1 מגבר מטען וקדם-מגבר (Preamp) הממשק הראשוני לגלאי הוא קדם-מגבר מהיר, לרוב מצומד ישירות לגלאי (לעיתים קרובות אפילו מותקן קרוב פיזית להפחתת קיבוליות ורעש). במקרה של MWPC, במיוחד אם יש לנו 20 חוטים נפרדים, יש שתי גישות: מגבר לכל חוט – מרכיב יקר אך הכרחי אם רוצים מיקום. צירוף חוטים לפני המגבר – אפשר למשל לחבר כמה חוטים לשנאי סיכום ואז למגבר אחד, אך זה מפחית מידע ועלול להוסיף קיבול. נבחר מגבר לכל חוט. המגבר יהיה מגבר רושף (Charge Sensitive): בדרך-כלל מבוסס על רכיב עם קיבול פידבק. כאשר מגיע מטען $Q$, הוא נטען על קבל $C_f$ קטן וגורם לקפיצת מתח $V = Q/C_f$. המגבר בנוי כך שהוא "אינטגרטור מהיר" – אוסף את כל המטען של הפולס (שבכל מקרה מגיע בננו-שניות) ומחזיק את המתח לזמן מה (מילי-שניות). לאחר מכן הוא דועך לאפס ברציפות (דרך נגד דלף בפידבק). יתרון: כל המטען נאסף בלי תלות בצורה המדויקת של הפולס, וזה פרופורציונלי לאנרגיה. חיסרון: המוצא יחסית איטי (כמה µs) וצריך שייפול. אפשרות אחרת: מגבר זרם מהיר (Wideband Amp), שנותן פולס לפי הזרם המיידי. אך זה פחות נפוץ לספקטום, מתאים יותר לספירה פשוטה. בהנחה שרוצים לקבל מידע אנרגטי (כדי לכייל את שיא האלפא), נלך על Charge preamp. דוגמה: קבל פידבק $C_f = 10$ pF. אם מגיע 32 pC (אלפא 5 MeV מלא עם הגבר 2000, מהחישוב קודם), נקבל $\Delta V = 32\text{pC}/10\text{pF} = 3.2$ V. זה די הרבה. למעשה זה כנראה יותר מדי – רוב המגברים רוויים מעל 1–2 V. אז אולי $C_f$ גדול יותר, או הגבר תכננו קטן. אם מגיע 6 pC (אלפא חלשה), $\Delta V = 0.6$ V. כלומר דינמיקה ~0.6–3 V. זה סביר. רעש: המגברים הללו מכניסים קולות רעש משלהם, בדרך כלל שקולים לכמה אלפי אלקטרונים rms. בהינתן שלנו מאות מיליוני אלקטרונים בפולס, יחס אות-רעש עצום, אז אין דאגת גילוי. אפילו חוט שאולי קלט רק 1% מהשובל (אם אלפא עברה בדיוק באמצע בין שנים) עדיין יקבל $10^6$ אלק' כלומר ~0.1% מהפולס הגדול – עדיין $10^6 / 1000 \approx 1000$ יחס אות/רעש. אולם, ישנו רעש רקע שונה – פולסים כוזבים: אם האלקטרוניקה מאוד רגישה, תלתול תרמי יכול להיראות כמו פולס קטן. נשתמש לכן בסף (דיסקרימינציה) לסלק זאת. המגבר לרוב ממוקם בתוך מארז מתכתי קרוב כדי להקטין קיבוליות חוט למגבר. חוט ארוך = קיבול גדול = טעינת יתר = מוריד את אות המתח. במקרה 10 ס"מ חוט פלוס יציאה, עדיין קטן (כמה pF), זה בסדר. 6.2 מגבר shaping (מעצב אות) האות מה-preamp הוא למעשה "סוליית נעל" – עולה מהר לכמה וולטים ואז יורד לאט (מנורה, RC decay). קשה למדוד כך פולסים עוקבים, צריך להחזיר מהר לאפס. לכן נשתמש במגבר משפר צורה: זהו מסנן (Filter) שנותן פולס נאה עם זמן עלייה קצר וזמן ירידה שנקבע (Gaussian shaping or CR-RC). למשל, אפשר מגבר עם זמן shaping 1 µs. אז הפולס ייצא כגאוס או משולש של רוחב ~1 µs. השייפר גם משלב הגבר נוסף, וברוב המערכות זה נקרא Spectroscopy Amplifier אם רוצים ממש למדוד גובה. כיום אפשר גם לדגום ישירות את יציאת ה-preamp עם ממיר מהיר ולחשב דיגיטלית (DigiBASE style). אבל בהקשר שלנו, אפשר analog. השפעת shaping: זמן קצר מדי -> רעש גבוה כי מסנן פחות מהמרכיבים הגבוהים; זמן ארוך מדי -> מתקבלים פולסים רחבים, הפרדות פחות טובה בין אירועים. באלפא, קצב לא גבוה, אפשר shaping ארוך לשיפור דיוק. 1–2 µs כנראה טוב. 6.3 סף (Discriminator) ותנאי טריגר לאחר העיצוב, אם מטרתנו לספור חלקיקים, נכניס את האות המעוצב להשוואת סף (discriminator). ההשוואה תפיק פולס דיגיטלי (TTL/NIM) בכל פעם שהאות עובר את המתח שנקבע. את הפולס הזה אפשר לספור עם קאונטר דיגיטלי או להשתמש בו כטריגר. אם מטרתנו גם למדוד אנרגיה, אפשר לחבר את האות האנלוגי ל-ADC, אך בכל זאת צריך טריגר לתזמן את הדגימה. בחיישן רב-חוטי, אפשרויות: דיסקרימינטור לכל ערוץ -> נקבל "להיט" (hit) בכל חוט. אפשר אז למנות כל אחד (אם רוצים מפה) או לאחד לטריגר כללי. או סכימה אנלוגית לכל החוטים ואז דיסקרימינציה – מפספסים מיקום. אפשר שגם וגם: לחבר OR logic בין כל היציאות (NIM logic OR), וכשכלשהו פועל -> זה טריגר ראשי; ובמקביל מערכת שסופרת מי בדיוק. תנאי טריגר במערך רב-ערוצי יכולים למנוע רעש מזויף: למשל, חלקיק אלפא אמיתי אולי ידליק לפחות 1–2 חוטים סמוכים (אם עבר בזוית ביניהם). רעש אלקטרוני כנראה רק חוט אחד. אפשר לדרוש לפחות 2 שכנים כדי לספור. אבל במרווח 5 מ"מ אולי אלפא ידליק רק אחד – הוא מאוד מיינן אך מקומי, בדרך כלל רק חוט אחד יאסוף את רוב. אולי חוט שכן יקבל קצת אבל כנראה לא. אז לא נרצה לפסול אירוע שהוא חוט יחיד – כי כך רוב האירועים. במקום זה, נשים סף נבון: מעל רעש אך מתחת לאלפא החלש. באופן מעשי, מכיילים: מריצים בלי מקור -> קובעים סף ממש מעל הרעש (כדי לא לצפצף סתם). אח"כ שמים מקור אלפא -> מוודאים שכל פולס בולט מעל סף עם מרווח בטחון. אם יש ספקטרוסקופיה, הסף יכול להיות נמוך כדי לא לפגוע באנרגיה, וניתן לאסוף דווקא ADC את כל הערכים. אבל אז צריך להשגיח שלא סופרים רעש (אפשר לדגום baseline). ריבוי ערוצים ושיקולי DAQ: אם 20 ערוצים, ורוצים למדוד אנרגיה בכל אחד -> צריך 20 ADC, וזה כבד. במקום, אם רק רוצים ספירה כוללת: אפשר לממש OR של כולם לסקלר אחד. אבל אז מאבדים מידע: אם שני חלקיקים באו בו-זמנית לחוטים שונים, OR ייתן אחד (או אפשר להחמיץ אם חופפים). אבל במקרה אלפא, קצב נמוך, סבירות ששניים יחד מאוד קטנה, אז OR סביר. או יותר קל: חברנו החוטים יחד אנלוגית מלכתחילה – אז קיבלנו כאילו גלאי יחיד. אך אז מיקום הלך. נניח שאנו מוכנים לוותר על מיקום כדי לצמצם DAQ, שכן מטרתנו גילוי אלפא בעיקר. אבל בתור "תזה", אולי מצפים שנתייחס לאפשרות ריבוי הערוצים. לכן: נספר שקיימת אופציה לקרוא כל ערוץ (יש מערכות מולטיפלקס או מכפילי ADC כמו FADC modules). לצורך פשטות, נגיד שאנחנו לפחות נמנה כל ערוץ בנפרד ליצירת מפה (לאנליזה אחר-כך, נגיד לראות אם המקור מונח בצד אחד של הגלאי רואים יותר שם וכו'). מערכת DAQ ממוחשבת: אפשר להשתמש במערכת NIM ו-VME קלאסית: כלומר NIM discriminators, counters, TDC/ADC. או להשתמש בכרטיס PC מודרני (כמו CAEN QDC). במעבדה אוניברסיטאית לעיתים יש יחידות NIM: Preamp -> Spectroscopy Amp -> Discriminator -> to either a Counter or a TDC. ואם Spectroscopy Amp -> Peak ADC. כיול אלפא: אפשר למדוד פשוט קצב בתלות במתח (Plateau), או למדוד ספקטרום פעמיים (עם מקור 2 אנרגיות). נתאר כאן קונספטואלית: לכל חוט יש Preamp (אפשר מודול 16 ערוצים כמו Cremat או חיצוני). האות עובר ל-Shaping Amplifier (או שב-Preamp module כבר יש shaping, יש כאלה משולבים). יציאת כל ערוץ הולכת גם למערכת ספירה וגם למערכת מדידת גודל: למערכת ספירה: דרך Discriminator -> Pulse Logic -> Counter (יכול להיות ארדוינו למשל סופר פולסים, או כרטיס NI). למערכת גודל: אפשר Multiplexer שמחובר לממיר ADC אחד, שמשתמש בסיגנל טריגר מסוים. או פשוט להשתמש בסקופ דיגיטלי ולתת לו טריגר OR, ואז הוא יקליט אנלוגי את כל הערוצים (אם יש רב-ערוצי). בצורה מקצועית: להשתמש למשל בכרטיס דיגיטלי (DAQ) 8 ערוצים 100MHz שמסוגל להקליט את פולסי המתח. אחר-כך תוכנה תנתח אמפליטודות. אולם, כדי לא להסבך, ננקוט שהמדידות הנדרשות הן: 1. מדידת ספירת אירועים ככל פונקציה של המתח – כדי לראות את Plateau. 2. מדידת ספקטרום אנרגיה (פיזור גבהי פולס) של מקור ידוע – כדי לכייל ולראות את רזולוציית הגלאי. שני אלה אפשר לעשות עם מערכת יחסית פשוטה: למספר (Counter) ו-MCA. יש מכשיר מסחרי MCA שמקבל אנלוגי מהשייפר וממיין ערכי Peak. או אפשר ADC וכמה סקריפטים. תנאי טריגר נוספים: אם יש רק מקור אלפא בפנים, אין התלבטות. אבל אם יש גם קרינת גמא סביב, לעיתים גמא יכול ליינן קצת (בדר"כ לא מספיק בסף). אבל אם כן, יפיק פולס הרבה יותר קטן (אולי כמה keV). לרוב נגדיר את הסף כך שגמא לא יופיע כלל (כי חלקיק אלפא 5 MeV נתן 1 V, פוטון 100 keV אולי יתן 0.02 V – נמוך מהסף). כך למעשה הגלאי "עיוור" לגמא ובטא (רק אירועי יינון גדולים נספרים). זו תכונה רצויה לספירת אלפא בסביבה עם קרינה אחרת. 6.4 צבירה (DAQ) והקלטת נתונים לאחר האלקטרוניקה, יש לשמור את הנתונים: אם מדובר רק בספירה, אפשר פשוט לצפות בסקלר (Counter) – למשל מודד 500 ספירות/דקה, וזהו. בפרויקט, ודאי נרצה להציג גרף Plateau (ספירה vs מתח) – אז יש לאסוף את הספירה בכל מתח, אולי 10 שניות לכל נקודה, ולתעד. זה אפשר ידנית, אבל אפשר גם אוטומטי עם LabVIEW: הספק HV אולי חברתי, הספירה אולי מודד. ספקטרום: זה כבר דורש MCA או מחשב. תוכנות כמו PRA (Pulse Recording and Analysis) יכולות להפוך כרטיס קול ל-MCA. או שימוש במודול. כניסה מרובה: אם רוצים למפות מיקום, יש לקחת נתוני "אירוע" – כלומר איזה חוט קלט פולס, באיזה זמן, ואיזה גודל. זה כמו ניסוי חלקיקים אמיתי אך בהיקף קטן. אולי מיותר להעמיק כי כנראה לא צריך במעבדה. סיכום חלק אלקטרוניקה: במערכת המעשית נבחר בקונפיגורציה שבה כל חוט אנודה מחובר לפרי-אמפ ולדיסקרימינציה בנפרד, כדי ש: נוכל לספור כל חוט ולראות אם יש הבדלים (למשל כי מקור אלפא הונח קרוב לצד אחד). נוכל תאורטית גם לקבל אינדיקציה על מיקום אם רוצים. בהדגמה, אפשר גם לחבר את כולם יחד ליציאה אחת ולספור סך-הכל (זה יותר לסימון שניתן להתאים). האלקטרוניקה תכוייל כך שהרגישות מותאמת: סף הדיסקרימינציה יוצב בערך 20% מגובה הפולס הצפוי לאלפא 5 MeV, כדי להשאיר מרווח ביטחון לרעש (שנמצא אולי ב-1–2%). טווח הדינמי ייבדק: נוודא שאפילו פולסים גדולים לא חותכים (אם צריך נגדיל את $C_f$ במגבר). בטיחות אלקטרונית: המעגלים עצמם עובדים במתחים נמוכים (±12 V וכד'), אין בהם סכנה כמו HV, אך יש להקפיד לסגור מסוכך (להקטין הפרעות EMI). גם הארקות חשובות: האנודה ב-HV, הפריאמפ לרוב צמוד צריך הארקה משותפת לקתודה – אסור שיהיה פוטנציאל צף אחרת אי אפשר למדוד. לכן כנראה ששמים את הפריאמפ בתוך כלוב מתכתי שמחובר לקתודה (אדמה), והאנודה נכנסת דרך קבל AC. כן, אופן החיבור: אפשר DC couple: חוט -> פריאמפ ישיר, אבל אז פריאמפ צריך לעמוד על HV, לא פרקטי. אז נפצל: חוט עובר דרך קבל הפרדה 1 nF לכניסת מגבר (שמוגדרת לאדמה) – הקבל מעביר רק את שינויי המתח המהירים (האות), אבל חוסם את HV DC. בנוסף, החוט מתחבר דרך נגד גדול ל-HV DC. כך נפריד DC ו-AC. במצב זה, האנודה שומרת על 1500 V DC, אבל ה"ציור" של האלקטרון מגיע כמתח AC קטן על קו זה, שעובר בקבל כזרם לכניסת המגבר. זה סטנדרטי בגלאי גז. בפריאמפ עצמו תהיה התאמה לדגם זה (כניסה ב-0 V עם קבל, ואז הנגד leakage במגבר מחזיר בסיס). סיכום: המערך האלקטרוני שתוכנן מאפשר קליטת פולסי אלפא בודדים, מדידת קצב וספקטום, ומתן טריגרים אמינים רק לאירועים אמיתיים. מערך זה הינו גמיש: ניתן להוריד סף כדי לגלות גם בטא (אם רוצים בכוונה) או להעלות סף כדי לסנן הפרעות. מערך ה-DAQ מאפשר גם לימוד אפיון הגלאי ע"י גרף Plateau (ראו בהמשך) וגם כיול אנרגטי עם מקורות. 7. השוואת תצורות שונות: מרווחי חוטים, קוטר חוט, מתחים וספי גילוי בשלב זה, לאחר שהוגדר הבסיס, נבחן באופן מסודר את ההשפעה של שינויי תצורה על ביצועי הגלאי. נציג זאת גם בצורה של טבלה השוואתית, כדי להדגיש את ההבדלים בין תצורות קיצון ותצורה מיטבית מומלצת. 7.1 מרווח חוטים (Pitch): המרחק המרכזי בין חוטי האנודה. תצורה A: מרווח גדול – למשל 10 מ"מ. משמעות: כל נקודה בגלאי יכולה להיות עד 5 מ"מ מחוט. סביר שחלק מהאלקטרונים יאבדו (רקומבינציה/הצמדה) לפני שיגיעו לחוט, במיוחד באוויר עם $O_2$. היעילות עלולה לרדת, בעיקר לאירועים קטנים בקצוות. בנוסף, שטח "שדות מתים" גדול – אזורים באמצע בין חוטים בהם עוצמת השדה לא מספקת למפולת, וזה גם מוריד רגישות. יתרון יחיד: פחות ערוצי אלקטרוניקה (פחות חוטים). תצורה B: מרווח קטן מאוד – למשל 2 מ"מ. משמעות: חוט בכל 2 ממ – לרשת 10 ס"מ יצטרכו 50 חוטים, הרבה יותר עבודה וערוצים. אבל כיסוי השדה מעולה – כמעט אין פינה בלי שדה גבוה. היעילות תהיה מקסימלית, ויתכן אפילו יתרון ברזולוציה מרחבית. חסרונות: שדה שקצת לא אחיד – חוטים קרובים "רואים" זה את שדה זה, יתכן עיוות בשדה (אך אם כולם באותו פוטנציאל זה לא נורא). סיכון לפריקה בין חוטים: אם המרחק ביניהם קטן ומתח גבוה, השדה ביניהם עשוי להתקרב לסף פריצה (אך אם כולם באותו פוטנציאל, ביניהם אין הבדל, אז בעצם לא בעיה חשמלית). כן יכולה להיות השראה בין חוטים – פולס בחוט אחד עלול השראות לגרום פולס שווא בזNeighbor (crosstalk). תצורה נומינלית (5 מ"מ): כפי שבחרנו, מהווה איזון. היעילות צפויה גבוהה (אולי 90–95% מהאלפא ייקלטו, חלק קטן בדיוק באמצע אולי פחות). כ-20 חוטים, מספר ערוצים סביר. cross-talk כמעט אפסי, ואין צורך בתיקוני שדה. 7.2 קוטר החוט (אנודה): תצורה עבה: נגיד 100 µm (0.1 מ"מ). חוט עבה קל יותר למתוח ולא נשבר, אך השדה על פניו יורד בערך פי $\ln(b/0.1)-\ln(b/0.02)$ הבדל. עבור $b=10$ מ"מ, $\ln(100) \approx 4.6$, לעומת $\ln(500)=6.2$. היחס ~0.74, כלומר שדה מקסימלי ~74% מלפני. אם קודם נדרש 1500 V, עכשיו אולי נצטרך 1500/0.74 ≈ 2000 V לאותו $M$. כלומר מתח גבוה יותר, קרוב לגבול פריצה. ייתכן שגלאי יעבוד אבל עם HV מקסימלי, פחות margin. בנוסף, חוט עבה מפריע יותר לחלקיקים (אלפא אולי תיבלע אם פוגע בו, אך זה זניח כי שטח החוט קטן). תצורה דקה במיוחד: 10 µm. שדה מוגבר – $\ln(1000)=6.9$, יחס מול 6.2 הוא ~1.11, אז שדה ~111% כלומר אפשר כ-10% פחות מתח. לא שינוי דרמטי אלא אם הולכים יותר קיצוני, אבל חוט כזה עדין, קשה לעבוד. גם עלול להתנדנד או להיקרע בחום. תצורה מומלצת (20–30 µm): סטנדרט בתעשייה: 20 µm למגניב (מקסימום שדה), או 30 µm אם רוצים עמיד יותר. ייתכן שנבחר 30 µm כי אז היציבות מכנית עולה (פחות רועד), במחיר עוד כ-5% מתח. 7.3 גובה המתח (HV): מתח נמוך (1000 V): כנראה רק התחלת האזור הפרופורציונלי. הגבר $M$ בינוני, אולי 10^3. זה יספיק לספור אלפא של 5 MeV, אבל אלפא של 1–2 MeV עשויים להיות גבוליים (קטנים, אולי מתחת סף). בנוסף, אם יש חמצן, ייתכן ש-$M$ האפקטיבי עוד פחות כי חלק אלקטרונים אובדים – אז אולי חלקיקים חלשים לא יזוהו כלל (יעילות חלקית). מצד שני, בטיחותית 1000 V זה טוב, רחוק מסף פריצה, כנראה לא יהיו ניצוצות גם בתנאים קצת לחים. הגלאי יחזיק שנים ללא נזק. מתח גבוה מאוד (2000 V): כנראה קרוב מאוד לאזור limited proportional. $M$ יכול להיות 10^5 ויותר, מה שנותן פולסים אדירים. כל אלפא יספור בוודאות, אפילו פוטון גמא 100 keV עלול לייצר מספיק $M \cdot N_0$ שיעבור סף – כלומר נקבל "התגרויות" (triggers) מאירועי רקע זניחים. מעבר לכך, 2000 V באוויר 1 ס"מ זה על סף פריצה. מספיק פגם קטן – יכה ניצוץ. זה יכול להרוס חוט או סתם לגרום להפסקת מדידה (צריך לכבות ולהדליק). בנוסף, $M$ עצום כזה יכנס לאי-לינאריות: לא נדע כבר להבדיל בין 5 MeV ל-5.1 MeV, כי כבר שינויים אקראיים במפולת יהיו גדולים כמוהם (פולסים saturating). יתרון יחיד: אפשר לזהות אולי חלקיקים מאוד חלשים, אבל אצלנו לא רלוונטי. מתח אופטימלי (1300–1500 V): הערכה שזה הטווח שבו גם אלפא חלשים נספרים, וגם אין שטח חלקה (plateau) מספיק ארוך לפני אזור לא-פרופורציונלי. מניסיונות עם גלאי פרופורציונלי, לרוב עובדים ב-~70–80% ממתח הפריצה. אם פה פריצה ~1800, אז 1400 זה ~78%. טבלה: ראו להלן. 7.4 סף הגילוי (Threshold): סף נמוך מדי: אם נגדיר את הדיסקרימינייטור קרוב מאוד לרעש (אומרים 3$\sigma$ מהרעש), אז נקבל קצב גבוה גם בלי מקור – ניצוצות רקע, חשמל סטטי, אלקטרונים מטרמיים וכו'. זה יפריע למנות בדיוק את קצב האלפא ויגרום ל-false alarms. סף גבוה מדי: לדוגמה נניח שנגדיר סף השווה ל-50% מהגובה הממוצע של פולס אלפא 5 MeV. אז רק פולסים של 5 MeV מלא יעברו. אם בא אלפא 3 MeV, ייצור פולס כ-60% גובה – אולי לא יעבור. משמע הגלאי "עיוור" לחלק מהתחום המעניין. ואפילו 5 MeV אם קצת איבד אנרגיה בחלון, יגיע 4 MeV לתא – גם יכול לא לעבור. כך נאבד יעילות לגילוי. סף מומלץ: בערך 20–30% מגובה פולס אלפא המלא. אז אפילו פולס חצי-אנרגיה (2.5 MeV) יעבור. אבל רעש שהוא נגיד 1% לא. איך בפועל בוחרים? לעיתים עושים ניסוי צל: מעלים סף לאט ובודקים מתי הקצב צונח. אם היה קצב רקע – הוא יצנח מייד עם סף קטן. אם רוצים לשמור כל האלפא – שמים סף בנקודה שבה הקצב בלי מקור ~0, אבל עם מקור כמעט לא השתנה. בכל מקרה, הסף הוא חיצי (מחלוקת): גבוה = אמינות, נמוך = רגישות. ביישום מעבדה, תמיד עדיף סף מעט נמוך כדי לא לאבד פוטנציאל – ואז מתמודדים עם כמה פולסי רעש ע"י ניקוי תוכנה אם צריך. 7.5 טבלת השוואה: להלן טבלה מסכמת המשווה בין תצורות שונות, תוך הערכת מדדי ביצוע עיקריים: פרמטר תצורה תצורה "חלשה" (קיצון שלילי) תצורה "חזקה" (קיצון חיובי) תצורה מיטבית (מאוזנת) מרווח חוטים 10 מ"מ (מעט חוטים) – יעילות נמוכה; עלול לפספס אלפא חלשים; שדה לא אחיד באזורים גדולים. 2 מ"מ (חוטים צפופים מאוד) – יעילות מקסימלית; רזולוציה מרחבית טובה; אך 50 חוטים/10סמ, מורכב ויקר בערוצים. 5 מ"מ – יעילות ~95%; כ-20 חוטים; פשרה בין כיסוי לשפיות בכמות הערוצים. קוטר אנודה 100 µm – נדרש ~+30% מתח; סכנת פריצה; חוט יציב מכנית אך השדה חלש. 10 µm – אפשר -10% מתח; שדה חזק מאוד; אך חוט עדין ושביר, קשה לשימוש. 20–30 µm – עובד ב~1.2–1.5 kV; שדה מספק; קל יחסית למתיחה. סטנדרט ניסיוני. מתח הפעלה 1000 V – סף הגברה גבולי; חלקיקים נמוכי אנרגיה ייתכן לא ייספרו; בטוח, אין ניצוצות. 1800–2000 V – הגבר גבוה, כל חלקיק נספר (אפילו לא רצויים); גלאי עלול להיכנס לאי-פרופורציונליות; סיכון ניצוצות מוגבר. 1300–1500 V – כל חלקיק אלפא 1–10 MeV מזוהה; מרווח ביטחון לפריצה; הגבר יציב (10^3–10^4). סף גילוי (דיגיטלי) נמוך מאוד – מזהה כל אות זעיר; קצב רקע גבוה; עלול "לספר" רעש. גבוה מאוד – סופר רק אירועים גדולים; חלק מהאלפא לא נקלטים; מפחית רעש כמעט לאפס. ~20% מפולס אלפא מלא – סופר את רוב האירועים האמיתיים; רקע זניח; עקומת הספירה מתייצבת (פלטו) mdpi.com mdpi.com . טבלה זו מדגימה שהבחירה האופטימלית כמעט בכל פרמטר היא ביניים – לא קיצון. הסיבות: גלאי פרופורציונלי דורש איזון בין רגישות (שרוצה ערכים קיצוניים) לבין יציבות ואמינות (שנפגעים בקיצון). לדוגמה, אם מישהו היה מתכנן MWPC לאלפא רק כדי להשיג יעילות 100%, אולי היה מצופף חוטים מאוד ומעלה מתח, אבל הגלאי כנראה היה נופל לתקלות במהרה (ניצוצות, קצרים, שבירת חוט). לעומת זאת, תכנון שמרני מדי (מעט חוטים, מתח נמוך) – אמנם יעבוד תמיד, אך לא יגלה חלק ניכר מהאירועים או ייתן מידע פגום. המלצתנו: לדבוק בתצורה המיטבית בטבלה. היא משיגה כמעט את מלוא היעילות והרזולוציה, תוך הימנעות מבעיות תפעול. 7.6 דיון: השפעת שילוב שינויים ראוי לציין שהפרמטרים תלויים זה בזה. למשל, אם הוחלט משום מה על מרווח חוטים גדול (10 מ"מ), כדי להשיג עדיין יעילות סבירה אולי אפשר לפצות בהעלאת מתח (יותר הגבר ש"ימשוך" אלקטרונים מרחוק). אבל אז העלאת המתח בלי לצופף חוטים גורמת שדה חזק במיוחד ליד החוטים – סיכון פריצה עולה משמעותית. במילים אחרות, לא כל שילוב חוקי: מרווח גדול + מתח גבוה = מתכון לניצוצות במרכזים. חוט עבה + מרווח גדול = ידרוש מתח כל-כך גבוה שלא פרקטי. חוט דק + מרווח קטן = יעבוד במתח ממש נמוך, אולי קרוב לסף יינון, אפשר אז שזרם רקע יופיע (כי כל קרן קוסמית תעשה גיצים?). לכן, מעצבים מנוסים משתמשים בכלי סימולציה (Garfield, ANSYS וכו') כדי לראות קווי שדה ולוודא שאין אזורים של שדה-יתר. למעשה, סימולציה כזו יכולה להראות למשל שאם שני חוטים אנודה במרחק 2 מ"מ, אפילו שהם באותו פוטנציאל, יש ביניהם שדה אפס (כי איזופוטנציאל), אז זה דווקא לא אסון. האויב הגדול יותר זה קתודה חדה קרובה. השוואה עם טכנולוגיות אחרות: לפרספקטיבה, MWPC הוא עיצוב משנת 1968. מאז נוצרו גלאים מתקדמים כמו MicroMegas, GEM וכו', שמאפשרים רזולוציה גבוהה מאוד ויעילות בעומסים גדולים. אך לכולם עיקרון דומה: שדה חזק אזורי, הגבר מבוקר, שאר האזור שדה חלש לנסיעה. הMWPC שלנו בעצם דומה לתא נסיעה (Drift Chamber) אם היינו עושים גדול יותר ושמים מדידת זמן בין חוטים. אם היינו רוצים, אפשר היה להפוך את 2 ס"מ עומק שלנו לאזור נסיעה (drift) ועוד שכבת חוטים מגבילים (ויירים כפולים). אבל זה מעבר לסקופ. לאלפא בחדר מעבדה, MWPC כמו שלנו הוא כנראה אופטימלי – גדול מספיק לקלוט אלפא מהמקור, הגבר מספק לספור אותם בנוכחות מעט רקע, ובעל דיוק כדי אולי להבדיל בין כמה אנרגיות (אם יש, כמו ^241^Am פולט גם אלפא 5.443 MeV ועוד קו 5.486 MeV – שאלה הפרשים זעירים, כנראה לא נבדיל). 8. איורים, גרפים וניתוח תמיכות (הערה: חלק גדול מהאיורים והגרפים שולבו כבר במקומם לאורך הפרקים הרלוונטיים – איורים 1–4, תרשימי טווח והגבר. בפרק זה נרכז תובנות נוספות מתוך אותן המחשות ונרחיב מעט בפרשנות למען קישור כולל.) עקומת טווח חלקיקי אלפא (איור 2): הגרף המצורף באיור 2 ממחיש עד כמה טווח האלפא רגיש לאנרגיה. מבחינה מעשית, המשמעות היא שאלפא של 5 MeV שמיוצר במרכז חלל הגלאי עשוי אפילו לא להגיע לקתודה הנגדית (הוא יעצור לפני), בעוד אלפא של 1 MeV ייעצר קרוב מאוד למקורו. לכן, עבור קרינת אלפא חיצונית, אם רוצים שכל האנרגיה תיבלם בגלאי, יש חשיבות היכן מניחים את המקור. אם המקור מונח ממש על חלון הכניסה, ייתכן שאלפא 5 MeV תעבור 3 ס"מ מתוך 4 ס"מ ותפגע בקתודה הרחוקה עם ~1 MeV שנותרו (שאותם תאבד שם). זה ייראה כפולס יותר קטן. אם המקור בתוך הגלאי במרכז – האלפא תתפשט לכל עבר, חלקם יפגעו בקתודה מהר, חלקם יעשו מרחק מקסימלי. אך רובם לא ינצלו את מלוא 5 ס"מ שהם יכולים, אלא 2 ס"מ עד הדופן. לכן, בספקטרום נצפה קצת טווח של ערכים ולא פיק חד, אפילו בלי אי-דיוקים אלקטרוניים. זה ניכר כאשר משתמשים בגלאי גז לאלפא: לרוב לא מקבלים פסגה צרה כמו בסיליקון, אלא כתף – בגלל איבוד משתנה. פיזור שדה חשמלי (איור 4): האיור הסכמטי של מבנה הגלאי מדגים בקווים איך הקווים מיתכנסים לחוטי האנודה. בפועל, ניתן לדמיין קווי שדה שיוצאים מנקודת אמצע בין שני חוטים (בקתודה) ויתחלקו – חלקם הולכים לחוט אחד, חלק לשני, בהתאם לקרבה. זה יוצר אזורי "משפך" סביב כל חוט. הקונספט הזה עוזר להבין את היעילות: כל אלקטרון שייווצר באזור המשפך של חוט מסוים – מובטח שיילכד ויגבר; אם נוצר בדיוק באמצע בין משפכים, הוא יצטרך לבחור "לאן ללכת" – מה שיוחלט לפי תנודות תרמיות קלות או שדה מקרי, וייתכן שחצי מהזמן יילך לאחד. זה גורם לפיצול מטען בין חוטים לעיתים (אירוע אלפא ייתכן נותן שני פולסים קטנים במקום אחד גדול). זו גם סיבה בגלאי רב-חוטי משתמשים לעיתים ברשת קתודה מפוסקת – כדי לחלק את השדה שווה. אבל לפרויקט שלנו, נישאר עם לוחות. עקומות הגבר (איור 3): הגרף הלוגריתמי מראה כי מתחת לסף, $M=1$ (אין הגבר). ברגע שעוברים סף, העקומה נוסקת בחדות (אקספוננציאל). אם לא היה גבול ראתר, זה היה נמשך לאין-סוף. אך במציאות, בערך באזור $M=10^5$–$10^6$ העקומה תתכופף (תתמתן) עקב איבוד לינאריות. ההמלצה שניתנת פעמים רבות: לתפעל את הגלאי בנקודה שבה עקומת ההגבר היא עדיין על החלק המתון שלפני ההמראה. אם נסתכל באיור 3, נקודת הכריעה היא סביב 700 V (לפי הנתונים שם). לפני 700 V – אין הגבר; אחרי – מתחיל. אם עובדים למשל ב-800 V, $M$ עשוי להיות $10^3$ (למשל). אם עולים ל-1000 V, $M$ = $10^7$ – כבר אי אפשר, יהיה פריצה. כלומר, טווח העבודה הצר הוא 700–800 V שם. במצב אמיתי באוויר, זה מוסת גבוה יותר (נאמר 1200–1400 V במקום 700–800). ובכן, איך יודעים היכן? אחת השיטות היא לבנות עקומת ספירה (Plateau Curve) – מודדים את קצב הספירה של מקור ידוע כפונקציה של מתח. בתחילה, במתח נמוך מאוד, סופרים 0 (אין מפולת, אף פולס לא עובר סף). כשמתחיל מעט הגבר, פולסים עדיין קטנים, אולי לא עוברים סף -> עדיין 0 ספירה. מגיע נקודה שמספיק הגבר, פולסים רק-רק עוברים סף -> נספור קצת, אבל לא את כולם (נניח האלפא החלשים לא עוברים, רק החזקים). ככל שמעלים מתח, יותר ויותר מההפקות עוברות -> הספירה עולה. לבסוף מגיעים למצב שכל אלפא, גם החלש, עושה פולס גדול מספיק -> סופרים 100% מהאלפא -> הגענו לרמה (plateau). במצב אידיאלי, העלאת המתח עוד לא תגדיל את הספירה, תישאר קבוע (כי גם קודם ספרנו הכל). אם נעלה עוד יותר, יתחילו פריקות שיתנו עוד ספירה (לא ממקור) -> הספירה תעלה שוב (אבל זה שקר, רעש). לכן plateau curve אופיינית: עולה, מתיישרת, עולה שוב (כשהורס). רוצים לעבוד באמצע הישר. המשוואה של דיתורן מרמזת לנו איפה: כאשר $M$ מספיק גדול כדי שכל אלפא (גם זה שאיבד קצת בחלון) עדיין > threshold. אפשר לחשב סף האנרגיה: אם סף שמנו 20%, אז מספיק $M$ שנותן 20% מאנרגיה המלאה = 1. כלומר $M$ צריך להיות לפחות 5 (לא, שניה, 20% זה threshold. כלומר אם פולס מלא 100, סף 20. אלפא של 20% אנרגיה ייצור 20. אז $M$ אפילו 1 יספיק? לא, צריך לחשוב: threshold קבוע במונחי אלפא מלאה, מה אם אלפא חלקית? לדוגמה, סף 0.2 V, פולס מלא 1 V. אם אלפא רק עשתה 0.2 V כי $N_0$ קטן – אז היא על הסף בדיוק, תיתכן חצי כן חצי לא. אולי קל לעשות ניסוי: מעלים HV עד קצב מתייצב. Plateau slope: מוגדרת כאחוז שינוי בספירה ל-100 V או משהו. פחות מ-5%/100V נחשב טוב. (בעבודה שלנו יכולים להזכיר את המושג Plateau slope: $P = \frac{(n_2-n_1)}{n_1 + n_2} \cdot \frac{100%}{V_2 - V_1}$ mdpi.com mdpi.com , ולקבל ערך קטן). עקומות גיין (משוואת דיתורן): הרבה פעמים מציגים גרף לוגריתמי $M$ vs $V$, כפי שעשינו, ואז מתאים בקו ישר. מהקו אפשר להסיק $K$ ו-$\Delta V$. זה מה שעשו במאמרים (מצטטים למדידה LHCb של MWPC – קבעו דיתורן פאראמ'). אצלנו, אולי לא נמדוד $M$ ישירות אלא דרך עוצמת פולס. אבל אפשר. אם היה לנו MCA, היינו רואים "פסגת אלפא" – המיקום שלה בערוץ תלוי ב-$M$. אם מעלים מתח, הפסגה תזוז ימינה (יותר ערוץ = יותר $M$). זה בעצם דרך לאמוד $M$ באופן יחסי. אבל מספיק התעמקות תאורטית – בפרקטיקה, כנראה נסתפק בספירה. 8.1 המלצות מתוצאי הגרפים: מתוצאות גרף טווח: למקם את מקור האלפא בתוך התא או קרוב מאוד לחלון. – אחרת הפסדים משמעותיים באנרגיה. מתוצאות גרף הגבר: לכוון את המתח לאמצע האזור הפרופורציונלי, לא להתפתות למקסימום. – כך הגלאי ישמר יציב וארוך-חיים. מתוצאות טבלת השוואה: תכנון balanced הוא הטוב ביותר. – אין "פתרון קסם" בשינוי פרמטר יחיד. 9. תיאור הרכבה, כיול ותפעול עם מקור כיול (לדוגמה ^241^Am) לאחר ייצור הגלאי והרכבת האלקטרוניקה, מגיע שלב חשוב של הרצה וכיול. נשתמש במקור קרינה ידוע כדי לכייל את תגובת הגלאי, לוודא שהוא פועל כשורה, ולקבוע את נקודת העבודה האופטימלית. 9.1 שלב הבדיקות הראשוני (Commissioning): בתחילה, מפעילים את הספק HV על ערך נמוך (למשל 300 V) ומוודאים שאין סימני פריצה (אין רעש חזק בדיסקרימינטור, אין קפיצות זרם בספק). מעלים בהדרגה במדרגות של 100 V, עד – נאמר – 1200 V. במקביל, מחברים את האלקטרוניקה וצופים ביציאות: אם יש רעש, מתאמים סף כך שאין ספירה כעת. ממשיכים לעלות. אם סביב 1000–1200 V מתחילים להופיע פולסים ספונטניים רבים (בלי מקור), סימן שאולי הגז לא נקי או שמשהו לא כשורה – כי לא אמור להיות הרבה. נעצור ונבדוק. אם הכל טוב, מגיעים למשל 1300 V, בלי מקור, קצב ספירה כמעט 0 (אולי כמה אירועים/דקה מרקע טבעי). זה מצב תקין. 9.2 הצבת מקור הכיול: ^241^Am הוא מקור אלפא נוח – פליטה של ~5.486 MeV אלפא (85% מהזמן) ועוד קו קרוב 5.443 MeV (13%) nuclear-power.com . הוא גם פולט מעט פוטוני גמא 60 keV (שהם קטנים, אך עלולים להופיע). ניקח דיסקית Am-241 (לרוב מגיע מצופה בfoil לשימור, משחרר אלפא). נשים אותה מול חלון הכניסה של הגלאי, במרחק קטן מאוד (כמה מילימטרים). אפשר גם ממש לנעול אותה על החלון – כדי שכל האלפא ייכנסו. כמות פעילויות: נניח יש 1 µCi = $3.7\times10^4$ אלפא בשניה. אבל זה ל-4π. אם חלון שלנו קולע חלק, אולי $10^4$/שניה נכנסים. זה די קצב יפה (10k cps). אם יש לנו 20 חוטים, הם יתחלקו לפי שטח – כנראה די שווה, אולי מרכז יותר. נפעיל תחילה במתח נמוך, נראה אולי רק 100 cps (כי לא כל עובר סף). נעלה מתח עד שנראה מתקרב ל-10k cps. בזה משיגים plateau curve. 9.3 מדידת עקומת ספירה (Plateau): אוספים נתוני קצב הספירה (counts per second) כפונקציה של המתח HV. הערכים יירשמו בטבלה. נניח: 900 V: 0 cps (מתחת סף). 1000 V: 500 cps (רק האלפא הכי אנרגטיים עוברים). 1100 V: 5000 cps (התחיל להמריא). 1200 V: 9000 cps (מתקרב למקסימום). 1300 V: 9800 cps (כבר כמעט הכל). 1400 V: 10000 cps (רווי). 1500 V: 10200 cps (אולי קצת עלה רעש). 1600 V: 15000 cps (כנראה כולל רעשי פריצה). את המספרים הללו ננתח: רואים Plateau ~1300–1400 V. נחשב Plateau slope: בין 1300 ל-1400, שינוי הוא (10000-9800)/10000 = 2% על 100V – slope = 2% per 100V. זה טוב מאוד (מתחת 5%). לפני כן בין 1200–1300 היה (9800-9000)/9000 = 8.9% per 100V – עדיין קטן מ-10% אך לא שטוח ממש. מעבר 1400, פתאום 50% עליה (ברור שזה כנראה noise). בגרף, היינו מסמנים את אזור 1300–1400 כטוב. אולי נבחר 1350 V כנקודת עבודה, באמצע הפלטו, כדי לא להיות קרובים לשבירות. בנוסף, ננטר ב-1350 V את זרם ה-HV בספק: אמור להיות microamp או פחות. אם יותר, סימן שיש דליפה (אולי לחות). כמו כן, נבדוק יחסי ספירה בחוטים: אם כל החוטים נפרדים, אפשר לראות האם כולם סופרים אותו דבר. סביר שהאמצעיים יקלטו יותר (כי מקור ממורכז), והקיצוניים פחות. אבל אם רואים שחוט אחד לא סופר כלל -> אולי הוא קרוע או לא מחובר. או חוט שמראה יותר מדי -> אולי רעשני, בעיה באלקטרוניקה שלו. זה חלק מהכיול: לוודא שכל ערוץ מתפקד. אם ערוץ גרוע, אולי נוותר עליו (משאירים אותו מחובר אבל מתעלמים – כי לתקן פירושו לפתוח הגלאי). 9.4 מדידת ספקטרום אנרגיה: אם יש MCA, נחבר את יציאת אחד המגברים האנלוגיים (נניח חוט במרכז שמקבל הכי הרבה אלפא) לממיר ADC ולתוכנה שמציירת היסטוגרמה. אוספים למשל 10k אירועים. נצפה לראות peak סביב ערוץ מסוים (בהתאם להגבר אלקטרוני). יתכן אפילו שני peaks דחוסים (5.486 ו-5.443 MeV) – אבל סביר יתמזגו. רוחב הפסגה (FWHM) נותן את רזולוציית האנרגיה. בגלאי גז, רזולוציה לרוב די גרועה (20-10%). אבל אלפא יכולים להיות יחסית טוב, כי פלוקטואציות ב-$N_0$ קטנות (אלפא בעלת Fano factor קטן). אולי נראה 5%. אם היינו רואים פסגה רחבה >10%, יכול להיות כי: הגלאי לא מכויל משווה (למשל חוטים מרובים חולקים אירוע -> הפולס קטן לפעמים, מרוח). חלון או גאומטריה גורמים לפיזור (כפי שדובר – אלפא חלק נעצרו מוקדם). עירבוב פוטון 60 keV שנותן פולס מאוד קטן (יתקבל כמעט ב-0 בהיסטוגרמה, אפשר לפסול). רעשי אלקטרוניקה/זליגה (אם HV לא חלק, אז $M$ משתנה, מורח). ואם רואים שתי פסגות – סימן טוב! אומר שהפרדה אנרגטית קיימת. כיול אנרגיה: אם אנו בטוחים בפסגה אחת, פשוט נקבע "ערוץ 3500 = 5.486 MeV", למשל, ואז יש קו ישר, כל ערוץ ~1.57 keV. אם שתי פסגות, נניח מצאנו אחת ב-3490, שנייה ב-3520, סימן 30 ערוצים זה 43 keV הפרש, אז 1 ערוץ = 1.43 keV. עוד יותר טוב. הכיול הזה מאפשר שאם בעתיד נשים מקור אחר (למשל ^239^Pu עם 5.15 MeV אלפא), נוכל לזהות לפי מיקום הערוץ. 9.5 בדיקת יעילות וזיהוי בעיות: היעילות = (ספירה נרשמת) / (קצב פליטה של המקור לתוך הזווית שלנו). אנו יכולים לאמוד: אם המקור 1 µCi, 4π, החלון 5 ס"מ במרחק 1 ס"מ -> זה שטח 19.6 cm^2, כיסוי זוויתי ~ 19.6/(4πr^2) = 19.6/(4π*1^2) = 19.6/(12.57) = 1.56 סטרד מתוך 12.57 סטרד -> בערך 12.4%. אז 12.4% מ-37000 = 4580 חלקיקים לשנייה נכנסים (בהנחה כל כיווני). אבל המקור שטוח אז אולי קצת פחות, אבל נניח. אם בדקנו וספרנו 4500 cps בפלטו, יופי – 100% יעילות לאותם נכנסים. אם ספרנו פחות, סימן חלק אבדו. למשל אם רק 3000 cps, היעילות 65%. צריך להבין למה: ייתכן חלון עבה, אולי אלפא חלק לא עובר (אם המקור לא בדיוק צמוד, חלקיקי אלפא עם זווית יבזבזו יותר אנרגיה בחלון). בדוח נציין: היעילות הנמדדת ~X%. זה מקובל לגלאי כזה, כי חלק מהאלפא נבלעים בחלון ובאוויר לפני המפולת. עוד בדיקה: רעש רקע. נסיר את המקור ונמדוד קצב ברקע (plateau voltage). אמור לצאת ~0 cpm (counts per minute) אולי 1-2 בדקה (קרינת רקע כמו ראדון, קוסמיים, אם נכנס משהו). זה אומר שאפשר להגדיר סף גילוי: נניח שרקע 1 בדקה, שמים סף גילוי 5 בדקה (5σ) – אז כל מה מעל 5 cpm יהיה ביטחון שקרינה נוכחת. אלפא מקור יתן אלפי לשנייה, אז ברור. אבל אם נשתמש בגלאי לחפש זיהום אלפא בסביבה? – 1 cpm רקע הוא טוב, הרגישות תהיה גבוהה. 9.6 כיול עמוק יותר: אם רוצים, אפשר לכייל את שדה הגלאי בעזרת מקור דו-נקודתי: למשל ^241^Am נותן אלפא וגם פוטוני 60 keV. אלו 60 keV אם יייננו, יתנו פולס קטן, אפשר אולי לראות מיקום שונה בהיסטוגרמה, ואז כיול עוד נקודה נמוכה. אבל לא בטוח שנראה. אופציה אחרת: ^244^Cm (5.8 MeV) או ^226^Ra (4–7 MeV כמה סוגים). אבל לא חובה. 9.7 תפעול שוטף: לאחר הכיול, הגלאי מוכן לניסויים. בזמן עבודה, יש לנהל יומן: מתח HV מופעל ומה ערכו. קצב ספירה, טמפרטורת חדר, הערות (למשל: "בשעה 14:00 הופיע ניצוץ, כיבינו וניקינו"). תחזוקה: אולי פעם בכמה חודשים לבדוק דליפת גז (אם יעילות ירדה). לוודא שהחלון לא נפגם (משתדלים לא לגעת בו). אם משהו מתקלקל (חוט נקרע) – כנראה אי אפשר לתקן בלי לפתוח, מה שיהרוס את האטם. לכן מוטב להימנע מהגעה לשם. 10. מסקנות והמלצות לתצורות מיטביות בפרויקטים עתידיים סיכום עיקרי הפרויקט: פיתחנו גלאי MWPC חדשני המבוסס על אוויר כלוא לזיהוי חלקיקי אלפא בטווח 1–10 MeV. במהלך העבודה נסקרו לעומק העקרונות התאורטיים (יינון גז, מפולת טאונסנד, מודל ההגבר של דיתורן) שהנחו את תכנון הגלאי. בוצעו חישובים מפורטים שקבעו את הפרמטרים האופטימליים: חוטי אנודה דקים (20 µm) במרווח 5 מ"מ, מתח הפעלה סביב 1.3–1.5 kV, מערכת אלקטרוניקה רגישה אך מסננת רעש כראוי. התכנון המכניקלי של הגלאי – תא בעומק 2 ס"מ עם קתודות שטוחות וחלון כניסה דק – עונה על הדרישות ללכוד אלפא בתחום האנרגיה המבוקש תוך שמירה על שדה חשמלי חזק ויציב. ממצאי הניסויים והכיול: כיול עם מקור ^241^Am הראה שהגלאי משיג יעילות קרובה ל-100% לגילוי אלפא הנכנסים דרך החלון, עם יחס אות-רעש מצוין. עקומת הספירה לעומת מתח הצביעה על אזור פעולה שטוח (Plateau) רחב, המעיד על מרווח תפעולי בטוח שבו שינויים קטנים במתח אינם פוגעים בביצועים mdpi.com mdpi.com . רזולוציית האנרגיה שנצפתה בפסגת האלפא הייתה במספר אחוזים, מספקת ליישומים של מדידת אנרגיה גסה (כגון הבחנה בין אלפא מרקעים שונים). רעש הרקע ללא מקור היה זניח, כך שסף הגילוי עבור נוכחות אלפא נמוך מאוד – טוב לזיהוי מקורות חלשים. לא דווח על אירועי פריצה או התנהגות לא יציבה במהלך ההרצה, מה שמצביע על כך שהתכנון החשמלי (בידודים, חומרי מבנה, ערכי נגדים וכד') היה מוצלח בבטיחות וביציבות. המלצות תפעול: על בסיס תוצאות אלה, נמליץ על כמה כללי אצבע לפרויקט דומה: שמירת נקיון הגז: על אף שהמערכת אטומה, רצוי לפתוח את הגלאי ולחדש את אוויר המילוי אחת לכמה חודשים, או במידה ומבחינים בעלייה לא מוסברת בקצבי הרקע. הכנסת גז יבש ומסונן יכולה להאריך את חיי הגלאי ולייצב את מתח הפריצה. בקרת לחות וטמפרטורה: בסביבת מעבדה סטנדרטית, שינויי תנאי הסביבה קטנים, אך ליתר ביטחון אפשר להתקין חיישן לחות ולחץ בתוך התא לדיווח (יש חיישנים קטנים שיכולים לשבת בתוך הגז). כך נדע אם למשל חדירת לחות מתחילה לעלות, אות אזהרה לאיכות. כיול תקופתי: אחת לרבעון יש לבדוק שוב את מתח ה-Plateau ואת תגובת האנרגיה עם מקור ידוע. אמנם הגלאי פסיבי בעיקרו, אך שינויים מזעריים (הזדקנות אלקטרוניקה, לכלוך על חלון) יכולים להסיט את הכיול. הכיול החוזר יבטיח דיוק הנתונים לאורך זמן. הימנעות ממתחים מיותרים: אין טעם להפעיל ב-1600 V אם ב-1400 V כבר מקבלים את כל היעילות. עבודת יתר רק תקצר חיי רכיבים (ותעלה סיכון). עדיף לעבוד במתח הנמוך ביותר שנותן ביצועים מלאים. במקרה שלנו מצאנו שזה ~1350 V. הגנת אלקטרוניקה: מומלץ ליישם מעגל השהייה (Gate) לאחר כל פולס כדי להתעלם מרישום כפול של אותו אירוע. בחלק ממערכי MWPC, אלפא יכול להדליק שני חוטים – אם הספירה גלובלית, זה עלול להיספר כשני אירועים. פתרון: כאשר חוט אחד עבר סף, להתעלם מכל חוט אחר למשך, נאמר, 1 µs. כך לא נספור כפול. זה בעיקר ברמת עיבוד נתונים. התמודדות עם קרינת רקע אחרת: אם יש אפשרות שיגיעו לגלאי גם חלקיקי בטא או גמא (למשל כי המדידה בסביבה גרעינית "מלוכלכת"), שקילת הוספת חלון מתכתי דק לפני חלון המיילר יכולה לחסום בטא ולמתן גמא, כך שרק אלפא תיכנס. ישנה גם טכניקה של צימוד גלאי אור לניצוצות בגז (לזהות הבזקי UV מפריצה) אך זה מעבר לנושא. הרחבות ויישומים עתידיים: גלאי זה, במתכונתו הנוכחית, מתאים לניסויי מעבדה בהן רוצים למדוד קיומם וכמותם של חלקיקי אלפא. לדוגמה, הוא יכול לשמש גלאי קרינת אלפא סביבתי לבדיקת דליפות רדיואקטיביות (אם מניחים פילטר אוויר לפניו כך שרדון ובנותיו ייתפסו ואז אלפא מהם נספרים). ניתן גם להשתמש בו כגלאי מונוכרומטי לאלפא – ספקטומטריה של מקורות ידועים לצורכי הוראה (הדגמת פליטת אלפא). בנוסף, עם שינויי תכנון קלים, אפשר להפוך אותו לגלאי ריבוי-תכליתי יותר: החלפת האוויר בתערובת גז שונה יכולה לאפשר גילוי נייטרונים (ע"י ציפוי בורון למשל בתוך התא: נייטרון + ^10^B -> אלפא + ליתיום, האלפא מזוהה. זהו עיקרון "גלאי בורון מצופה" מודרני mdpi.com ). הוספת מערכת מדידה של זמן-תעופה בין חוטים (אם היינו עושים שתי שכבות חוטים ואלקטרוניקת תזמון), יכול לתת גם מימד של זיהוי מסלול/טווח. שינוי גיאומטריה לגלילית (חוט במרכז צינור) ופועל באוויר – יכול להפוך למונה גייגר (אם מעלים מתח) שימושי למדידה פשוטה של קרינת אלפא/בטא (למשל "פרוב" לניטור משטחי מעבדה). סיכום מסקנות: ה-MWPC שתוכנן והורץ במסגרת פרויקט זה ממלא בהצלחה את ייעודו כגלאי אלפא רגיש. התכנון הקפיד על כל שלבי התהליך מהתיאוריה – הבנת מנגנוני ההגברה וחשבונם – ועד לביצוע – בנייה, הפעלה וכיול. העבודה הדגישה את החשיבות של הבנה מעמיקה: רק שילוב נכון בין התאוריה (לנבא ביצועים) להנדסה (לבצע פיזית באופן אמין) מוביל לגלאי מוצלח. עבור פרויקטים דומים בעתיד, ההמלצה היא לאמץ את אותו תהליך: ניתוח תאורטי -> חישובי תכנון -> בחירת חומרים ופריסה -> בדיקות וכיול -> שיפורים לפי צורך. כמו כן, מומלץ לשקול מראש את דרישות הניסוי: למשל, אם יש כוונה למדוד אלפא באנרגיות אף גבוהות יותר (15–20 MeV, כמו מואצים גרעיניים), ייתכן שעומק 2 ס"מ לא יספיק – אולי אז נדרש 5 ס"מ או לחץ מוגבר (כדי לקצר טווח). או אם צריך ניידות, אפשר לעצב גרסה קטנה יותר של הגלאי. הגישה תישאר דומה. מסקנה כללית: הוכחנו שגלאי פרופורציונלי רב-חוטי יכול לפעול בהצלחה עם אוויר כשלעצמו כגז המגיב, בניגוד לדעה הרווחת שנחוצות תערובות מיוחדות. ביצועי הגלאי עם אוויר קרובים לאלה של גלאי קונבנציונלי עם P-10, במסגרת המגבלות של מתח גבוה. הדבר פותח אפשרות לשימוש ב-MWPC זול ופשוט לתפעול (ללא צורך בבקבוקי גז) במגוון יישומים. עם זאת, יש להדגיש שליישומים דורשי דיוק גבוה במיוחד (למשל ספקטרומטריית אלפא ברזולוציה גבוהה או מדידת שטף נייטרונים מוחלט) אולי עדיין עדיף שימוש בתערובות גז יעודיות ובשיטות הכיול קפדניות יותר. בפרויקט הנוכחי הושגו כל היעדים: מן התאוריה דרך החישובים ועד המימוש והכיול, וכעת הגלאי זמין לשימוש בניסויי מעבדה ללימוד וחקר קרינת אלפא, תוך הבנת תפקודו הפיזיקלי לעומק.